\documentclass[letterpaper,12pt]{article}
\usepackage{amsmath,amsfonts,amsthm,amssymb,graphicx,mathtools,tikz,hyperref}
\usepackage{esvect,esint, subcaption, mathrsfs,comment, quiver, faktor}
\usepackage[onehalfspacing]{setspace}
\usepackage[ddmmyyyy]{datetime}
\renewcommand{\dateseparator}{.}


% place figures and tables where it seems like they should be
\usepackage{float}
\floatplacement{figure}{H}
\floatplacement{table}{H}

% automatically center figures/tables
\makeatletter
\g@addto@macro\@floatboxreset{\centering}
\makeatother

\usepackage{tocvsec2}
\usepackage[bookmarksdepth=subsection]{hyperref}
\usepackage{bookmark}
\usepackage[margin=1in]{geometry}
\usepackage{authblk}
\usepackage{titling}
\setlength{\droptitle}{-1in}

\newcommand{\A}{\mathbb{A}}
\newcommand{\B}{\mathbb{B}}
\newcommand{\C}{\mathbb{C}}
\newcommand{\D}{\mathbb{D}}
\newcommand{\E}{\mathbb{E}}
\newcommand{\F}{\mathbb{F}}
\newcommand{\G}{\mathbb{G}}
\newcommand{\Hb}{\mathbb{H}} %
\newcommand{\I}{\mathbb{I}}
\newcommand{\J}{\mathbb{J}}
\newcommand{\K}{\mathbb{K}}
\newcommand{\Lb}{\mathbb{L}} %
\newcommand{\M}{\mathbb{M}}
\newcommand{\N}{\mathbb{N}}
\newcommand{\Ob}{\mathbb{O}} %
\newcommand{\Pb}{\mathbb{P}} % 
\newcommand{\Q}{\mathbb{Q}}
\newcommand{\R}{\mathbb{R}}
\newcommand{\Sb}{\mathbb{S}} % 
\newcommand{\T}{\mathbb{T}}
\newcommand{\U}{\mathbb{U}}
\newcommand{\V}{\mathbb{V}}
\newcommand{\W}{\mathbb{W}}
\newcommand{\X}{\mathbb{X}}
\newcommand{\Y}{\mathbb{Y}}
\newcommand{\Z}{\mathbb{Z}}
\newcommand{\Ac}{\mathcal{A}}
\newcommand{\Bc}{\mathcal{B}}
\newcommand{\Cc}{\mathcal{C}}
\newcommand{\Dc}{\mathcal{D}}
\newcommand{\Ec}{\mathcal{E}}
\newcommand{\Fc}{\mathcal{F}}
\newcommand{\Gc}{\mathcal{G}}
\newcommand{\Hc}{\mathcal{H}} %
\newcommand{\Ic}{\mathcal{I}}
\newcommand{\Jc}{\mathcal{J}}
\newcommand{\Kc}{\mathcal{K}}
\newcommand{\Lc}{\mathcal{L}} %
\newcommand{\Mc}{\mathcal{M}}
\newcommand{\Nc}{\mathcal{N}}
\newcommand{\Oc}{\mathcal{O}} %
\newcommand{\Pc}{\mathcal{P}} % 
\newcommand{\Qc}{\mathcal{Q}}
\newcommand{\Rc}{\mathcal{R}}
\newcommand{\Sc}{\mathcal{S}} % 
\newcommand{\Tc}{\mathcal{T}}
\newcommand{\Uc}{\mathcal{U}}
\newcommand{\Vc}{\mathcal{V}}
\newcommand{\Wc}{\mathcal{W}}
\newcommand{\Xc}{\mathcal{X}}
\newcommand{\Yc}{\mathcal{Y}}
\newcommand{\Zc}{\mathcal{Z}}

\renewcommand\qedsymbol{$\blacksquare$}
\renewcommand{\bar}[1]{\overline{#1}}
\renewcommand{\epsilon}{\varepsilon}

\newcommand{\ita}[1]{\textit{#1}}
\newcommand{\com}[2]{#1\backslash#2}
\newcommand{\oneton}{\{1,2,3,...,n\}}
\newcommand\idea[1]{\begin{gather*}#1\end{gather*}}
\newcommand\ef{\ita{f} }
\newcommand\eff{\ita{f}}
\newcommand\proofs[1]{\begin{proof}#1\end{proof}}
\renewcommand*{\proofname}{demostraci\'on}
\renewcommand*{\figurename}{Figura}
\newcommand\inv[1]{#1^{-1}}
\newcommand\setb[1]{\{#1\}}
\newcommand\en{\ita{n }}
\newcommand{\vbrack}[1]{\langle #1\rangle}
\DeclareMathOperator{\rank}{rank}
\DeclareMathOperator{\ord}{ord}
\DeclareMathOperator{\Span}{span}
\DeclareMathOperator{\Int}{Int}
\DeclareMathOperator{\cl}{cl}
\DeclareMathOperator{\diam}{diam}

\theoremstyle{plain}
\newtheorem{theorem}{Teorema}
\newtheorem{axiom}{Axioma}
\newtheorem{lemma}[theorem]{Lema}
\newtheorem{proposition}[theorem]{Proposici\'on}
\newtheorem{postulate}{Postulado}
\newtheorem*{corollary}{Corolario}

\theoremstyle{definition}
\newtheorem*{definition}{Definici\'on}
\newtheorem{conjecture}{Conjetura}
\newtheorem{example}{Ejemplo}
\newtheorem*{homework}{Tarea}

\theoremstyle{remark}
\newtheorem*{claim}{Reclamo}
\newtheorem*{note}{Nota}

\pretitle{
    \begin{center}
        \fontsize{14pt}{1em}
        \bfseries\selectfont
            MATE6551-0U1 \\
            Prof. Ivan Cardona Torres \\
            13.00 - 14.20 \\
            CNL-A-225 \\
            \vspace{14pt}
}

\posttitle{\end{center}}
\preauthor{\begin{center}\fontsize{12pt}{1em}\selectfont}
\postauthor{\end{center}}
\predate{\begin{center}\fontsize{12pt}{1em}\selectfont}
\postdate{\end{center}}

% section heading formatting
\renewcommand{\thesection}{\Roman{section}.}

\usepackage{sectsty}
\sectionfont{\centering\fontsize{12pt}{1em}\selectfont}

\title{Linear Algebra}

%%%%%%%%%%%%%%%%%%%%%%%%%%%%%%%%%%%%%%%%%%%%%%%%%%%%%%%%%%%%%%%%%%%%%%%%%%%%%%%%
% Your names go here (one should be underlined)
%%%%%%%%%%%%%%%%%%%%%%%%%%%%%%%%%%%%%%%%%%%%%%%%%%%%%%%%%%%%%%%%%%%%%%%%%%%%%%%%
\author{Alec Zabel-Mena}
\affil{Universidad de Puerto Rico, Recinto de Rio Piedras}

\begin{document}
\maketitle
\section*{Lectura 1: Grupos y Subgrupos}

\begin{definition}
    Sea $G$ un conjunto no vacio junto a una operaci\'on binaria  $\cdot$.
    Decimos que el par  $(G,\cdot)$ es un \textbf{grupo} si:
    \begin{enumerate}
        \item[(1)] $a \cdot b \in G$ para $a,b \in G$.

        \item[(2)] $a \cdot (b \cdot c)= (a \cdot b) \cdot c$, para $a,b,c \in G$

        \item[(3)] Existe un $e \in G$ tal que  $a \cdot e=e \cdot a=a$ para
            toda  $a \in G$.

        \item [(4)] Para toda $a \in G$, existe una  $\inv{a} \in G$ tal que $a
            \cdot \inv{a}=\inv{a} \cdot a=e$.
    \end{enumerate}
    Si $a \cdot b = b \cdot a$ para toda  $a,b \in G$, entoces decimos que  $G$
    es un grupo  \textbf{Abeliano}.
\end{definition}

\begin{example}\label{}
    \begin{enumerate}
        \item[(1)] Los naturales $\N$ junto a la multiplicaci\'on es  satisface
            los primeros tres axiomas, pero no es un grupo.

        \item[(2)] El grupo mas peque\~no es el conjunto $\{e\}$, que denotamos
            como $\vbrack{e}$. $\vbrack{e}$ es un grupo Abeliano.

        \item[(3)] Los enteros $\Z$ junto con adici\'on  $+$ forma un grupo
            Abeliano.

        \item[(4)] El conjunto $GL(n,\R)$ de matrices $n \times n$ con entradas
            reales, nosingular forman un grupo con respecto a multiplicaci\'on
            de matrices. $GL(n,\R)$ no es un grupo Abeliano.

        \item[(5)] Sea $S$ cualquier conjunto y  $A(S)$ el conjunto de todas las
            permutaciones de de elementos de $S$. Entonces  $A(S)$ es un grupo
            no Abeliano con respecto a composicion de funci\'ones, $\circ$. Si
            $S$ tiene  $n$ elementos, entonces  $A(S)=S_n$.
    \end{enumerate}
\end{example}

\begin{definition}
    Sea $G$ un grupo. El \textbf{orden} de un grupo es su cardinalidad, y
    escribimos $\ord{G}=|G|$. Decimos que $G$ es  \textbf{finito} si $\ord{G}$
    es finito; de lo contrario, $G$ es  \textbf{infinito}.
\end{definition}

\begin{definition}
    Sea $G$ un grupo, y  $a \in G$. El  \textbf{orden} de $a$, denotado
    $\ord{a}$, es el menor entero positivo $n$ tal que  $a^n=e$ y escribimos
    $\ord{a}-n$. Si tal $n$ no existe, entonces decimos que $a$ es de orden
    \textbf{infinita}, y decimos que $a$ es un elemento  \textbf{torsi\'on}.
\end{definition}

\begin{example}\label{}
    \begin{enumerate}
        \item[(1)] Considera $\C^*=\com{\C}{0}$, entonces $\C^*$ tiene orden
            infinita, note que si $\exp(\frac{2i\pi}{5}) \in \C^*$, entonces
            $\alpha \neq 1$, para $j \neq 1,2,3,4$, pero $\alpha^5=1$. Entonces
             $\ord{\alpha}=5$.

         \item[(2)] Considere $A \in GL(6,\R)$ con la forma
             \begin{equation*}
               A=\begin{pmatrix}
                     0 & 0 & 0 & 0 & 1 & 0 \\
                     0 & 0 & 0 & 0 & 0 & 1 \\
                     1 & 0 & 0 & 0 & 0 & 0 \\
                     0 & 1 & 0 & 0 & 0 & 0 \\
                     0 & 0 & 1 & 0 & 0 & 0 \\
                     0 & 0 & 0 & 1 & 0 & 0 \\
                 \end{pmatrix}
             \end{equation*}
            Entonces
             \begin{equation*}
               A^3=\begin{pmatrix}
                     1 & 0 & 0 & 0 & 0 & 0 \\
                     0 & 1 & 0 & 0 & 0 & 0 \\
                     0 & 0 & 1 & 0 & 0 & 0 \\
                     0 & 0 & 0 & 1 & 0 & 0 \\
                     0 & 0 & 0 & 0 & 1 & 0 \\
                     0 & 0 & 0 & 0 & 0 & 1 \\
                 \end{pmatrix}
             \end{equation*}
             entonces, $A^3=I$.

         \item[(3)] En $\R^*=\com{\R}{0}$, $\R^*$ es infinito, y  $\ord{2}$ es
             infinito.
    \end{enumerate}
\end{example}

\begin{definition}
    Sea $G$ un grupo y  $H \subseteq G$ no vacio. Entonces decimos que  $H$ es
    un  \textbf{subgrupo} de $G$ si  $H$ es un grupo bajo la misma opearaci\'on
    de  $G$. Escribimos  $H \leq G$.
\end{definition}

\begin{example}\label{}
    \begin{enumerate}
        \item[(1)] Considere $GL(n,\R)$ y sea $SL(n,\R)$ los elementos $A \in
           GL(n,\R)$ tales que $\det{A}=1$. Entonces $SL(n,\R) \leq GL(n,\R)$.

       \item[(2)] Sea $C(\R)$ el conjunto de todas las funciones continuas sobre
           $\R$. Entonces  $C(\R)$ es un grupo bajo la suma de funci\'ones $+$.
           Sea $C^1(\R)$ el conjunto primer difirenciable continua de funciones
           sobre $\R$. Observe lo siguiente:
           \begin{enumerate}
               \item[(a)] $(f+g)'=f'+g'$
               \item[(b)] $f'+(g+h)'=(f+g)'+h'$.
               \item[(c)] $c'=0$, entonces $0 \in C^1(\R)$
               \item[(d)] $f'-f'=-f'+f'=0$.
           \end{enumerate}
           Como todos los funciones de arriba tambien son continuas, vemos que
           $C^1(\R) \leq C(\R)$.
    \end{enumerate}
\end{example}

\begin{lemma}\label{}
    Sea $G$ un grupo y  $H \subseteq G$ no vacio. Si tenemos que $ab \in H$,
    implicat que $a\inv{b} \in H$, entonces $H \leq G$.
\end{lemma}
\begin{proof}
    Como $H \neq \emptyset$, sea  $a \in H$. Entonces $a\inv{a}=e \in H$. Luego,
    tambien tenemos qie $e\inv{a}=\inv{a} \in H$. Finalmente, tenemos que si $b
    \in H$, entonces  $a\inv{b} \in H$, por lo tanto $\inv{b} \in H$, entonces
    $a\inv{(\inv{b})}=ab \in H$.
\end{proof}

\begin{example}\label{}
    \begin{enumerate}
        \item[(1)] Considere a los enteros pares $2\Z$. Sean  $2n,2m \in 2\Z$.
            Noten que $2n-2m=2(n-m) \in 2\Z$. Entonces $2\Z \leq \Z$.

        \item[(2)] Si $G$ es un grupo, entonces  $\vbrack{e}$ y $G$ son
            subgrupos de  $G$. Llamamos a $\vbrack{e}$ el grupo
            \textbf{trivial}.

        \item[(3)] Si $G$ es un grupo, y  $a \in G$, entonces el conjunto
            $\vbrack{a}=\{a^j : j \in \Z\}$ es un subgrupo de $G$, llamado el
            \textbf{subgrupo generado por $a$}.

        \item[(4)] Si $G$ es un grupo, y  $a \in G$, entonces  $C(a)=\{g \in G :
            ag=ga\}$ y $Z(G)=\{g \in G : ag=ga \text{ para toda } a \in G\}$ son
            subgrupos. Nota que $Z(G)=\bigcap{C(a)}$. Llamamos a $C(a)$ el
            \textbf{cnetralizador} de $a$ y  $Z(G)$ el \textbf{centro} de $G$.

        \item[(5)] Sea $G$ un grupo y  $H \leq G$, y sea  $a \in G$, entonces
        $\inv{a}Ha \leq G$. Llamamos a $\inv{a}Ha$ el \textbf{conjugado} de $H$
         \textbf{con respecto} a $a$.
    \end{enumerate}
\end{example}

\begin{definition}
    Suponga que $G$ y  $H$ son grupos. Un mapa  $\phi:G \rightarrow H$ se llama
    un \textbf{homomorphismo} si para toda $a,b \in G$,
    $\phi(ab)=\phi(a)\phi(b)$. Si $\phi$ es  $1-1$ y sobre, entonces lo llamamos
    un  \textbf{isomorphismo}. Si $\phi$ es un isomorphismo, y  $G=H$, entonces
    llamamos a  $\phi$ un  \textbf{automorphismo}.
\end{definition}

\section*{Lecture 2: Smallest Infinite Cardinality}

\begin{definition}
    A set $A$ is  \textbf{countably infinite} if there exists a $1-1$ map $f:\N
    \rightarrow A$ of $\N$ onto  $A$.
\end{definition}
\begin{remark}
    That is we can put $A$ into countable order and represent it as an
    increasing sequence $A=\{a_n\}_{n \in \N}$ where each $a_n$ is distinct for
    all  $n$.
\end{remark}

\begin{example}\label{}
    Both $\N$ and  $2\N$ are countable. Consider the identity map  $i:\N
    \rightarrow \N$ and the map $f:\N \rightarrow 2\N$ defined by $n \rightarrow
    2n$.
\end{example}

\begin{theorem}\label{thm_2.1}
    A set $A$ is finite if, and only if there is no  $1-1$ map of $A$ onto a
    subset of itself.
\end{theorem}
\begin{corollary}
    Every infinite set countains a countable subset.
\end{corollary}

\begin{theorem}\label{}
    A countable union of sets is countable.
\end{theorem}
\begin{proof}
    Let $A$ and  $B$ be nonempty countable sets. Then there exists nonrepeating
    increaseing sequences  $A=\{a_n\}$ and $B=\{b_m\}$. Then take $A \cup
    B=\{a_n,b_m\}$, and delete any repeated members. Then we can say that $A
    \cup B=\{c_k\}$ where $c_k=a_k$ for  $k$ odd, and  $c_k=b_k$ for  $k$ even.
    Then  $A \cup B$ is countable.

    Now suppose that  $\{A_n\}$ is a collection of countable sets. Then we have
    the following:
    \begin{align*}
        A_1     &=      a_{11} \ a_{12} \ \dots \ a_{1n} \ \dots \\
        A_2     &=      a_{21} \ a_{22} \ \dots \ a_{2n} \ \dots \\
        A_3     &=      a_{31} \ a_{32} \ \dots \ a_{3n} \ \dots \\
                \vdots                              \\
        A_n     &=      a_{n1} \ a_{n2} \ \dots \ a_{nn} \ \dots \\
                \vdots                                \\
    \end{align*}
    Construct the set:
    \begin{equation*}
        A=\{a_{11},a_{12},a_{21},a_{13},a_{22},a_{31}, \dots\}
    \end{equation*}
    Then $A$ is countable and $A=\bigcup{A_n}$.
\end{proof}

\begin{example}\label{}
    \begin{enumerate}
        \item[(1)] $-\N$ is countable, then  $-\N \cup \N=\Z$ is also countable,
            which makes $\Z$ countable.

        \item[(2)]  Consider the rational numbers $\Q=\{\frac{p}{q} : p,q \in
            \Z\}$. Let $\frac{\Z}{n}=\{\frac{p}{n} : p \in \Z\}$ for some $n \in
            \com{\N}{\{0\}}$. Then notice that  $\frac{\Z}{n}$ is countable and
            that $\Q=\bigcup{\frac{\Z}{n}}$, which makes $\Q$ countable as well.
    \end{enumerate}
\end{example}

\begin{lemma}\label{}
    Let $A$ and  $B$ be two countably finite sets. Then  $A \times B$ is
    countable.
\end{lemma}
\begin{proof}
    Take $A \times B=(\bigcup{A_n}) \times \{b_n\}$, where $b_n \in B$.
\end{proof}

\begin{example}\label{}
    \begin{enumerate}
        \item[(1)] Let $P$ be the set of all polynomials with integer
            coefficients. Let  $P_i$ be the set of all polynomials of degree
            $\deg=i$. Then we have the following:
            \begin{align*}
                P_0     &\simeq     \Z      \\
                P_1     &\simeq     \Z \times \Z        \\
                        \vdots              \\
                P_n     &\simeq     \Z \times \Z \times \dots \times \Z        \\
                        \vdots                      \\
            \end{align*}
            Then we have that $P \simeq \bigcup{P_i}$. Since each $P_i$ is
            countable, so is  $\bigcup{P_i}$, so by isomorphism, $P$ must also
            be countable.

        \item[(2)] Let $\Ac=\{x \in \C : \text{ there exists a polynomial with
                integer coefficients } p \text{ such that } p(x)=0\}$ be the set
                of all algebraic numbers over $\C$. Then we have  $\Q \subseteq
                \Ac$, since the polynomial $z^2+1=0$ has complex roots. $\Ac$
                is countable.
    \end{enumerate}
\end{example}

\begin{theorem}\label{}
    Let $A$ be a set, then  $|A|<|2^A|$.
\end{theorem}
\begin{proof}
    We have that by definition, $|A|>|2^A|$ is impossible, so suppose that
    $|A|=|2^A|$. Then there exists a $1-1$ map  $f:A \rightarrow 2^A$ of $A$
    onto its powerset. Then we have that for each  $a \in A$,  $f(a) \subseteq
    A$. Now, let $A_0=\{x \in A : x \notin f(x)\}$. Since $f$ is onto, there is
    an  $x_0 \in A$ with $f(x_0)=A_0$. Now, if $x \in f(x_0)$, we get that $x
    \in A_0$, which means that $x_0 \notin f(x_0)$. This cannot happen; so it
    must be that $x_0 \notin f(x_0)$, but then we get that $x_0 \in A_0$, which
    forces $x_0 \in f(x_0)$. A contradiction! Therefore it must be that
    $|A|<|2^A|$
\end{proof}

\begin{example}\label{}
    $|\N|<|2^\N|$
\end{example}

Now consider the set of real numbers $\R$. We have the following theorems.

\begin{theorem}\label{}
    $\R$ is a field.
\end{theorem}

\begin{theorem}\label{}
    $\R$ is a totally ordered set.
\end{theorem}
\begin{corollary}
    The following are true for all $a,b,c \in \R$.
    \begin{enumerate}
        \item[(1)] $a>b$ impl ies that $a+c>b+c$.

        \item[(2)] $a>b$ and  $c>0$ implies  $ac>bc$.

        \item[(3)] For all $a>0$ and  $b>0$, there exists an  $n \in \N$ such
            that  $an>b$  (Property of Archimedes).
    \end{enumerate}
\end{corollary}

\begin{theorem}\label{}
    $\R$ has the least upperbound property. That is, if  $\{a_n\}$ is an
    increasing sequence of real numbers bounded above by $b$, then  $\sup{\{a_n\}}
    \in \R$.
\end{theorem}

\end{document}
