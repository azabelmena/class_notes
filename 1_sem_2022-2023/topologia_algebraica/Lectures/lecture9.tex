\section*{Lectura 9: Caminos y Conexidad por Caminos.}

\begin{definition}
    Un \textbf{camino} en un espacio topologica $X$ es una mapa continua
    $f[0,1] \xrightarrow{} X$ tal que $f(0)=a$ y $f(1)=b$ para algunos $a,b \in
    X$. Decimos que  $a$ es el  \textbf{punto inicial}, y $b$ el  \textbf{punto
    final}. Decimos que $f$ empieza  \textbf{desde} $a$  \textbf{hacia} b.
\end{definition}

\begin{definition}
    Un espacio topologico $X$ es  \textbf{conexo por caminos} s\'i para toda
    $a,b \in X$, existe un camino  $f:[0,1] \xrightarrow{} X$ desde $a$ hacia
    $b$.
\end{definition}

\begin{theorem}\label{9.18}
    S\'i un espacio topologico $X$ es conexo por caminos, entonces es conexo.
\end{theorem}
\begin{proof}
    Sea $X$ no conexo. Entonces existe una separacion de  $X$ en conjuntos
    abiertos $U$ y  $V$ disjuntos; osea, $X=U \cup V$. Sea $a \in U$ y $b \in
    V$. S\'i $f [0,1] \xrightarrow{} X$ es un camino desde $a$ hacia $b$,
    entonces existe una separaci\'on de la imagen $f([0,1])$. Pero $[0,1]$ es un
    conjunto conexo, por lo tanto, $f([0,1])$ tambien tiene que ser conexo. Esto
    es una contradicci\'on.
\end{proof}

\begin{example}\label{}
    No todo los conjuntos conexos son conexos por camino.
    \begin{enumerate}
        \item[(1)] Considera la curva seno del topologo.

        \item[(2)] Considera el remolino del topologo.
    \end{enumerate}
\end{example}

\begin{theorem}\label{thm_9.19}
    S\'i $X$ es un espacio topologico, entonces la relaci\'on  $\sim$ dado por
    $a \sim b$ s\'i y solo s\'i existe un camino desde  $a$ hacia  $b$. Entonces
     $\sim$ es una realci\'on de equivalencia.
\end{theorem}
\begin{proof}
    El camino constante $[0,1] \xrightarrow{} X$ dado por $x \xrightarrow{} a$
    hace que $a \sim a$. Entonces, considere  $a \sim b$. Entonces existe un
    camino  $f:[0,1] \xrightarrow{} X$ dado por $f:0 \xrightarrow{} a$ y $f:1
    \xrightarrow{} b$. Defina entonces $g=f(1-t)$. $g$ es continua por
    composicion, y  $g:0 \xrightarrow{} b$ y $g:1 \xrightarrow{} a$. As\'i que
    $b \sim a$.

    Finalmente, sea  $a \sim b$ y  $b \sim c$. Entonces existen caminos
    $f:[0,1] \xrightarrow{} X$ y $g:[0,1] \xrightarrow{} X$ dados por $f:0
    \xrightarrow{} a, 1 \xrightarrow{} b$ y $g:0 \xrightarrow{} b, 1
    \xrightarrow{} c$. Defina la mapa $h:[0,1] \xrightarrow{} X$ dado por
    \begin{equation*}
        h(t)=\begin{cases}
            f(2t), \text{ s\'i } 0 \leq t \leq \frac{1}{2}  \\
            g(2t-1), \text{ s\'i } \frac{1}{2} \leq t \leq 1  \\
        \end{cases}
    \end{equation*}
    Por la teorema del empaste, $h$ es continua. Mas a\'un,  $h:0 \xrightarrow{}
    f(0)=a, 1 \xrightarrow{} g(1)=c$. As\'i que $a \sime c$.
\end{proof}

\begin{definition}
    Se llama las clases de equivalencia del espacio topologico $X$ bajo conexidad
    de caminos los \textbf{componentes de caminos} de $X$. Escribimos
    $\pi_0(X)=\faktor{X}{\sim}$ el conjunto de los componentes de camino.
    Dado $f:X \xrightarrow{} Y$, definimos $\pi_0(f):\pi_0(X) \xrightarrow{}
    \pi_0(Y)$ la mapa que envia un conjunto de caminos, $C$ de $X$ a los unicos
    compnentes de caminos de $Y$, que contienen a $f(C)$.
\end{definition}

\begin{theorem}\label{thm_9.20}
    $\pi_0:\Top \xrightarrow{} \Conj$ es un funtor.
\end{theorem}
\begin{proof}
    Sea $1_X:X \xrightarrow{} X$, y sea $\pi_0(x)=\{X_\alpha\}$ donde $X_\alpha$
    es un componente de camino. Entonces $\pi_0(1_X):X_\alpha \xrightarrow{}
    X_\beta$, as\'i que $X_\alpha \subseteq X_\beta$. Como  $X_\alpha$ y
    $X_\beta$ son clases de equivalencia, entonces  $X_\alpha=X_\beta$, es
    decir,  $\alpha=\beta$. Por lo tanto  $\pi_0(1_X)=1_{\pi_0(X)}$.

    Ahora sean $f:X \xrightarrow{} Y$y $g:Y \xrightarrow{} Z$ continuas. Sean
    $\pi_0(X)=\{X_\alpha\}$, $\pi_0(Y)=\{Y_\beta\}$, y $\pi_0(Z)=\{\Z_\gamma\}$
    los conjuntos de componentes de caminos de,  $X$,  $Y$, y  $Z$
    respectivamente. Considere  $X_\alpha$ y  $Z_\gamma$ tales que  $\pi_0(g
    \circ f)(X_\alpha)=Z_\gamma$. Entonces $Z_\gamma$ contiene a $g \circ
    f(X_\alpha)=g(f(X_\alpha))$. Sea $Y_\beta$ el comoponente que contiene a
    $f(X_\alpha)$, es decir, $\pi_0(f)(X_\alpha)=Y_\beta$. Entonces $Y_\gamma$
    es el unico tal componente, as\'i que vemos que
    $g(f(X_\alpha)) \subseteq g(Y_\beta) \subseteq Z_\gamma$, y $Z_\gamma$ es el
    unico componente que contienes a  $Y_\beta$, y a  $X_\alpha$. Por lo tanto,
    tenemos que $\pi_0(g \circ f)=\pi_0(g) \circ \pi_0(f)$. Esto hace a $\pi_0$
    un funtor en $\Top.$
\end{proof}
\begin{corollary}
    S\'i  $f \simeq g$, son del mismo tipo de homotopia, entonces  $\pi_0(f) = \pi_0(g)$.
\end{corollary}
\begin{proof}
    Suponga que $f \simeq g$ atraves de una homotop\'ia $F$. Sea $C$ un
    componente por caminos de  $X$, entonces,  $C \times I$ es conexo por
    caminos. Entonces, tenemos que $F(C \times I)$ es continua, por lo tanto,
    $F(C \times I)$ es conexo por caminos. Ahora, tenemos que $f(C)=F(C \times 0)
    \subseteq F(C \times I)$; de igaul forma, $g(C) \subseteq F(C \times 1)
    \subseteq F(C \times I)$. As\'i que el componente por caminos unico
    conteniendo a  $F(C \times I)$ contiene a $F(C)$ y a $g(C)$. Por lo tanto,
    $\pi_0(f)=\pi_0(g)$.
\end{proof}
\begin{corollary}
    S\'i dos espacios tienen el mismo tipo de homotopia, entonces tienen el
    mismo numero de componenetes por caminos.
\end{corollary}
\begin{proof}
    Suponga, que $f:X \xrightarrow{} Y$ y $g:Y \xrightarrow{} X$ son continuas,
    y satisfacen $g \circ f=1_X$ y  $f \circ g=1_Y$. Aplicando  $\pi_0$, venos
    que $\pi_0(g \circ f)=\pi_0(g) \circ \pi_0(f)=1_{pi_0(X)}$ y
    $\pi_0(f \circ g)=\pi_0(f) \circ \pi_0(g)=1_{pi_0(Y)}$. Esto implica que
    $\pi_0(g)$ es sobre, y que $\pi_0(f)$ es 1--1 en la categoria $\Conj$. De
    igaul manera, vemos que $\pi_0(f)$ es sobre, y que $\pi_0(g)$ es 1--1, as\'i
    que $\pi_0(f)$ y $\pi_0(g)$ son ambos 1--1 y sobre. As\'i que $\pi_0(X)$ y
    $\pi_0(Y)$ son equienumerables, as\'i que tienen el mismo numero de
    componentes por caminos.
\end{proof}
