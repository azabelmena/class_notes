\section*{Lectura 11: El Grupo Fundamental.}

\begin{definition}
    Sea $X$ un espacio topologico y  $f:[0,1] \xrightarrow{} X$ y $g:[0,1]
    \xrightarrow{} X$ caminos en $X$ con  $f(1)=g(0)$. Definimos el
    \textbf{producto de caminos} de ser la operacion $\ast$ que lleva  $(f,g)
    \xrightarrow{} f \ast g$ tal que
    \begin{equation*}
        f \ast g(t)=\begin{cases}
                f(2t), & \text{ s\'i } 0 \leq t \leq \frac{1}{2}    \\
                g(2t-1), & \text{ \s'i } \frac{1}{2} \leq t \leq 1  \\
            \end{cases}
    \end{equation*}
\end{definition}

\begin{lemma}\label{11.31}
    El producto de caminos es una mapa continua.
\end{lemma}
\begin{proof}
    Esto sigue del teorema del empaste.
\end{proof}
\begin{corollary}
    El producto de caminos es un camino.
\end{corollary}

\begin{definition}
    Sea $X$ un espacio topologico, y  $A$ subespacio de  $X$. Sean  $f_0:X
    \xrightarrow{} Y$ y $f_1:X \xrightarrow{} Y$ mapas continuas con
    $f_0|_A=f_1|_A$. Decimos que $f_0$ es \textbf{relativamente homotopico} a
    $f_1$, relativo a $A$ s\'i existe una mapa continua $F:X \times [0,1]
    \xrightarrow{} Y$ tal que $F:f_0 \simeq f_1$ y $F(a,t)=f_0(a)=f_1(a)$ para
    todo $a \in A$. Escribimos  $f_0 \simeq f_1 \rel{A}$ y llamamos a $F:f_0
    \simeq f_1 \rel{A}$ la \textbf{homotopia relativa} entre $f_0$ y $f_1$.
\end{definition}

\begin{example}\label{}
    Sea $X$ un espacio topologico y  $A=\emptyset$. Entonces la relacion  $f_0
    \simeq f_1 \rel{A}$ es nada mas que la homotopia usual $f_0 \simeq f_1$.
    Llamamos a este homotopia relativa la \textbf{homotopia libre}.
\end{example}

\begin{lemma}\label{11.32}
    La relacion $\simeq_A$ de homotopia relativa es una relacion de
    equivalencia.
\end{lemma}

\begin{definition}
    Sea $\partial{I}$ el borde de $I=[0,1]$. Llamaos las clases de equivalencias
    de la homotopia relativa $\simeq_{\partial{I}}$ \textbf{clases de caminos}.
    S\'i $f$ es un camino, denotamos el clase de caminos de  $f$ por  $[f]$.
\end{definition}

\begin{theorem}\label{11.32}
    Suponga que $f_0,f_1$ y $g_0,g_1$ son caminos en un espacio topologico $X$
    tales que  $f_0 \simeq f_1 \rel{\partial{I}}$ y $g_0 \simeq g_1
    \rel{\partial{I}}$. S\'i $f_0(1)=g_0(0)$ y $f_1(1)=g_1(0)$, entonces
    $f_0 \ast g_0 \simeq f_1 \ast g_1 \rel{\partial{I}}$.
\end{theorem}
\begin{proof}
    Sean $F:f_0 \simeq f_1 \rel{\partail{I}}$ y $G:g_0 \simeq g_1
    \rel{\partail{I}}$ homotopias relativas entres  $f_0$ y $f_1$, y $g_0$ y
    $g_1$. Defina la mapa $H:[0,1] \times [0,1] \xrightarrow{} Y$ dado por
    \begin{equation*}
     H(s,t)=\begin{cases}
                 F(2s,t), & \text{ s\'i } 0 \leq s \leq \frac{1}{2} \\
                 G(2s-1,t), & \text{ s\'i } \frac{1}{2} \leq s \leq 1   \\
            \end{cases}
    \end{equation*}
    Por el teorema del empaste, vemos que $H$ es continua, ademas vemos que
    $H(0,t)=F(0,t)=f_0 \ast g_0(t)$, y $H(1,t)=G(1,t)=f_1 \ast g_1(t)$. As\'i
    que $H:f_0 \ast g_0 \simeq f_1 \ast g_1 \rel{\partial{I}}$ es una homotopia
    relativa.
\end{proof}
\begin{corollary}
    $[f_0 \ast g_0]=[f_1 \ast g_1]$.
\end{corollary}
