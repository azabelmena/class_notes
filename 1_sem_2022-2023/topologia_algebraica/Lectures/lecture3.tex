\section*{Lectura 3: Categor\'ias y Funtores.}

\begin{definition}
    Definimos una \textbf{clase} de ser una colecci\'on de objetos tal que s\'i
    $T$ y  $A$ son clases, entonces  $A \notin T$.
\end{definition}

\begin{definition}
    Una \textbf{categor\'ia} $\Cc$ es un clase de objetos denotados por
    $\obj{\Cc}$ junto a una colecci\'on de conjuntos $\Hom{(X,Y)}$,  para
    cualquieras $X,Y \in \obj{\Cc}$, de \textbf{morfismos} de $X$ hac\'ia $Y$,
    cuyas elementos estan denotados $f:X \rightarrow Y$ \'o $X \xrightarrow{f}
    Y$, y una operac\i\'on binaria $\circ:\Hom{(X,Y)} \times \Hom{(Y,Z)}
    \rightarrow \Hom{(X,Z)}$ llamado \textbf{composici\'on} tal que si $f:X
    \rightarrow Y$ y $g:Y \rightarrow Z$ son morfismos, entonces $g \circ f:X
    \rightarrow Z$ es un morfismo y:
    \begin{enumerate}
        \item[(1)] $\Hom{(X,Y)}$ y $\Hom{(A,B)}$ son disjuntas.

        \item[(2)] La composici\'on $\circ$ es associativa s\'i esta definido.
            Es decir, sy  $g \circ (f \circ h)$ \'o $(g \circ g) \circ h$
            existen en $\Hom{(X,Y)}$, entonces $g \circ (f \circ h)=(g \circ f)
            \circ g$.

        \item[(3)] $\Hom{(X,X)}$ no es vac\'io y existe al menos un morfismo
            $1_X:X \rightarrow X$, llamado la \textbf{identidad} de $X$, tal que
            $1_X \circ f=f$ y  $g \circ 1_X=g$ para morfismos  $f:X \rightarrow
            Y$ y $g:Z \rightarrow X$, para cualquieras objetos $X,Y,Z \in
            \obj{\Cc}$
    \end{enumerate}
\end{definition}

\begin{definition}
    Sea $\Cc$ una categor\'ia. Se llama el conjunto $\Mc_\Cc$  \textbf{los
    morfismos de la categor\'ia} donde $\Mc_\Cc$ es la union de todos los
    conjuntos $\Hom{(X,Y)}$ para todos $X,Y \in \obj{\Cc}$.
\end{definition}

\begin{figure}[h]
    \centering
    \[\begin{tikzcd}
	X &&& Y &&& Z
	\arrow["f", from=1-1, to=1-4]
	\arrow["g", from=1-4, to=1-7]
	\arrow["{g \circ f}"', curve={height=30pt}, from=1-1, to=1-7]
\end{tikzcd}\]
    \caption{Un ejemplo de composici\'on de morfismos de una categor\'ia}
    \label{fig_6}
\end{figure}

\begin{figure}[h]
    \centering
    \[\begin{tikzcd}
	X & X &&& Y & Y
	\arrow[curve={height=12pt}, from=1-2, to=1-5]
	\arrow[curve={height=-12pt}, from=1-2, to=1-5]
	\arrow[from=1-2, to=1-5]
	\arrow["{1_X}", from=1-2, to=1-1]
	\arrow["{1_Y}", from=1-5, to=1-6]
	\arrow[curve={height=24pt}, from=1-5, to=1-2]
	\arrow[curve={height=-24pt}, from=1-5, to=1-2]
\end{tikzcd}\]
    \caption{Morfismos entre dos objetos $X$ y  $Y$ de una categor\'ia
    incluyendo las identidades de $X$ y $Y$}
    \label{fig_7}
\end{figure}

\begin{example}\label{}
    \begin{enumerate}
        \item[(1)] Considere la categor\'ia $\Cc=\Conj$, donde  $\boj{\Cc}$ es
            la clase de todo los conjuntos. Los morfismos de $\Cc$ son
            funci\'ones  $f:X \xrightarrow{} Y$ de un conjunto $X$ hac\'ia un
            conjunto  $Y$.

        \item[(2)] Sea $\Cc=\Top$ la categor\'ia de espacios topol\'ogicos,
            donde  $\obj{\Cc}$ es la colecci\'on de todas las espacios
            topol\'ogicos. Los morfismos de $\Top$ son funci\'ones continuas
            entre espacios topologicos. Es decir, $\Hom{(X,Y)}=\{f : f:X
            \xrightarrow{} Y \text{ es continua}\}$. La composici\'on de
            morfismos es la composicion de funciones usual.

        \item[(3)] Sea $\Cc=\Grp$ la categor\'ia de grupos, cuyas objetos son
            todo los grupos. Entonces los morfismos de  $\Grp$ estan definido
            por los conjuntos  $\Hom{(G,H)}=\{\phi : \phi:G \xrightarrow{} H
            \text{ es un homomrfismo}\}$. La composici\'on de morfismos es la
            composicion de funciones usual.
    \end{enumerate}
\end{example}

\begin{definition}
    Sean $\Cc$ y  $\Ac$ categor\'ias con  $\obj{\Cc} \subseteq \obj{\Ac}$.
    Decimos que $\Cc$ es una \textbf{subcategor\'ia} de $\Ac$ s\'i
    $\Hom_\Cc{(X,Y)} \subseteq \Hom_\Ac{(X,Y)}$ para todo $X,Y \in \obj{\Cc}$ ty
    la composici\'on de $\Cc$ es la misma de  $\Ac$.
\end{definition}

\begin{example}\label{}
    \begin{enumerate}
        \item[(1)] Tenemos que $\Top$ y  $\Grp$ son subcategor\'ias de
            $\Conj$.
        \item[(2)] La categor\'ia $\Top^2$ de pares topologicos tiene como objetos
            son todas pares  $(X,A)$, donde $X$ es un espacio topol\'ogico y $A
            \subseteq X$ es subsepacio de $X$. Los morfismos de $\Top^2$, para
            pare topologicos  $(X,Y)$ y $(Y,B)$, son las funciones continuas
            $f:X \xrightarrow{} Y$ donde $f(A) \subseteq B$ es subespacio de $B$.

        \item[(3)] La categor\'ia $\Top^*$ de pares topologicos  $(X,a)$, donde
            $a$ es un punto en  $X$ es una subcategor\'ia de  $\Top^2$.
    \end{enumerate}
\end{example}

\begin{definition}
    Sea $\Cc$ una categor\'ia. Una \textbf{diagrama} de objetos y morfismos en
    $\Cc$ es un grafo dirigido cuya cunjunto de vertices es subconjunto de
    $\obj{\Cc}$ y cuyas aristas son morfismos entre esos vertices. Decimos que
    una diagrama es \textbf{commutativo} si para cualquieras vertices $A,B,C,D$
    en la diagrama, y cualquier morfismos  $f:A \xrightarrow{} B$, $i:C
    \xrightarrow{} D$, $h:A \xrightarrow{} C$, y $g:B \xrightarrow{} D$, tenemos
    que $g \circ f=i \circ h$.
\end{definition}

\begin{example}\label{}
    Las figuras \ref{fig_6} y \ref{fig_7} son ejemplos de diagramas de objetos y
    morfismos en una categori\'ia.
\end{example}

\begin{figure}[h]
    \centering
    \[\begin{tikzcd}
	A &&&&& B \\
	\\
	\\
	C &&&&& D
	\arrow["f", from=1-1, to=1-6]
	\arrow["i", from=4-1, to=4-6]
	\arrow["g", from=1-6, to=4-6]
	\arrow["h"', from=1-1, to=4-1]
	\arrow["{g \circ f=i \circ h}", dashed, from=1-1, to=4-6]
\end{tikzcd}\]
    \caption{Un diagrama commutativa entre objetos y morfismos de una
    categor\'ia.}
    \label{fig_8}
\end{figure}
