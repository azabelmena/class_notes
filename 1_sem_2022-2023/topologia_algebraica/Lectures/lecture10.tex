\section*{Lectura 10: Simplejas.}

\begin{definition}
    Un subconjunto $A \subseteq \R^n$ es  \textbf{af\'in} s\'i para todo $x,y
    \in A$, la recta  $l$ que pasa por  $x$ y  $y$ esta contenido en  $A$.
\end{definition}

\begin{lemma}\label{10.23}
    S\'i $A \subseteq \R^n$ es af\'in, entonces es convexo.
\end{lemma}

\begin{example}\label{}
    Los conjuntos $\emptyset$ y $\{x\}$, con $x \in \R^n$ son af\'in.
\end{example}

\begin{theorem}\label{10.24}
    S\'i $\{X\alpha\}$ es una colecci\'on de conjuntos afines en $\R^n$,
    entonces el interseccion de todos es afin.
\end{theorem}
\begin{proof}
    Sea $X=\bigcap{X_\alpha}$ con $X_\alpha \in \R^n$ af\'in. Sean  $x,y \in X$
    y  $l(x,y)$ la recta que pasa por $x$ y  $y$. Entonces  $l(x,y) \in
    X_\alpha$ para todo $\alpha$ lo cual hace  $l(x,y) \in X$. Por lo tanto $X$
    es af\'in.
\end{proof}
\begin{corollary}
    S\'i $\{X\alpha\}$ es una colecci\'on de conjuntos convexos en $\R^n$,
    entonces el interseccion de todos es convexos.
\end{corollary}

\begin{definition}
    Sea $X \subseteq \R^n$. Llamamos al interseccion de todos los conjuntos
    afines conteniendo a $X$ el  \textbf{casco af\'in} abarcado por $X$. De
    igual forma, la interseccion de todos conjuntos convexos conteniendo a  $X$
    se llama el \textbf{casco convexo} abarcado $X$. En ambos casos, denotamos
    el casco afin o convexo como $[X]$. S\'i $x_0, \dots, x_m \in \R^n$,
    escribimos $[\{x_0, \dots, x_m\}]=[x_0, \dots, x_m]$.
\end{definition}

\begin{definition}
    Una \textbf{combianci\'on af\'in} de puntos $x_0, \dots, x_m \in \R^n$ es un
    punto $x$ tal que:
    \begin{equation*}
        x=t_0x_0+\dots+t_mx_m
    \end{equation*}
    donde $\sum{t_i}=1$. Una \textbf{combinaci\'on convexo} es una combinacion
    afin donde $y_i \geq 0$ para todo  $1 \leq i \leq m$.
\end{definition}

\begin{theorem}\label{10.25}
    S\'i $x_0, \dots, x_m \in \R^$, entonces $[x_0, \dots, x_m]$ el casco
    convexo abarcado por estos puntos es el conjunto de toda las combinaciones
    convexas de $x_0, \dots, x_m$.
\end{theorem}
\begin{proof}
    Sea $S$ el conjunto de todas las combinaciones convexas de  $x_0, \dots,
    x_m$. Entonces $[x_0, \dots, x_m] \subseteq S$; pues, sea $t_j=1$ y $t_i=0$
    para todo  $i \neq j$, entonces  $x_j \in S$, as \'i que $\{x_0, \dots,
    x_m\} \subseteq S$. Ahora, sea $\alpha=\sum{a_ix_i}$ y $\beta=\sum{b_ix_i}$
    donde $a_i,b_i \geq 0$ y  $\sum{a_i}=\sum{b_i}=1$. Nota que
    \begin{equation*}
        t\alpha+(1-t)\beta=\sum{(ta_i+(1-t)b_i)x_i}
    \end{equation*}
    mas aun, $\sum{ta_i+(1-t)b_i}=1$ y que $ta_i+(1-t)b_i \geq 0$ para todo $0
    \leq i \leq m$. As\'i que  $S$ es convexo.

    Ahora, sea $X$ convexo con  $\{x_0, \dots, x_m\} \subseteq X$ por induccion
    en $m$, s\'i  $m=0$, $S=\{x_0\}$ lo que hace $S \subseteq X$. Ahora,
    considere para $m \geq 0$. Sea  $t_i \geq 0$ y  $\sum{t_i}=1$ tal que
    $x=\sum{t_ix_i}$. Si $t_0=1$, entonces $p=\sum{t_ix_i}=x_0 \in X$. Ahora, si
    $t_0 \neq 1$, considere la combinacion
    \begin{equation*}
        q=(\frac{t_1}{1-t_0})x_1+\dots+(\frac{t_m}{1-t_0})x_m
    \end{equation*}
    Entonces, vemos $q$ es una combinacion convexa, y por hipotesis,  $q \in X$.
    Entonces la combinacion  $x=t_0x_0+(1-t_0)q \in X$, lo que hace $X$ convexo.
    Ahora, nota que  $[x_0, \dots, x_m]$ es convexo y contiene $\{x_0, \dots,
    x_m\}$, por lo tanto $S \subseteq [x_0, \dots, x_m]$.
\end{proof}

\begin{definition}
    Una colecci\'on de puntos $x_0, \dots, x_m \in \R^n$ son \textbf{af\'in
    independiente} si la colecci\'on $x_1-x_0, \dots x_m-x_0$ es linealmente
    independiente en $\R^n$ como espacio vectorial.
\end{definition}

\begin{example}\label{}
    $\emptyset$ y $\{x_0\}$ son afin independientes en $\R^n$.
\end{example}

\begin{theorem}\label{10.26}
    Las siguientes enunciados son equivalentes para todo $x_0, \dotx, x_m \in
    \R^n$:
    \begin{enumerate}
        \item[(1)] $x_0, \dots, x_m$ son af\'in independiente.

        \item[(2)] S\'i $a_0, \dots, a_m \in \R^n$ tales que $\sum{a_ix_i}=0$ y
            $\sum{a_i}=0$, entonces $a_0=\dots=a_m=0$.

        \item[(3)] S\'i  $A$ es abarcado afinmente por  $x_0, \dots, x_m$,
            entonces cada $x \in A$ tiene una representacion unica como
            combinacion afin de  $x_0, \dots, x_m$.
    \end{enumerate}
\end{theorem}
\begin{proof}
    Suponga, primero, que $x_0, \dots, x_m$ son af\'in independiente. Ahora, sea
    $a_0, \dots, a_m \in \R^n$ tales que $\sum{a_i}=0$, y que $\sum{a_ix_i}=0$.
    Entonces vemos que
    \begin{equation*}
        \sum{a_ix_i}=\sum{a_x_i-0 \cdot x_0}=\sum{(a_ix_i-x_0\sum{a_i})}
        =\sum{a_i(x_i-x_0)}=0
    \end{equation*}
    Como $x_0, \dot, x_m$ son af\'in independientes, entonces $x_1-x_0, \dots,
    x_m-x_0$ son linealmente independiente, lo cual implica que
    $a_0=\dots=a_m=0$.

    Ahora, suponga que el segundo enunciado sea cierto. Sea  $A$ abarcado por
    $x_0, \dots, x_m$, y suponga que $x \in A$ tal que  $x=\sum{a_ix_i}$ y
    $\sum{a_i}=1$. Sea tambien $x=\sum{b_ix_i}$ donde $\sum{b_i}=1$. Entonces
    tenemos que $\sum{a_ix_i}=\sum{b_ix_i}$ por lo tanto
    $\sum{(a_i-b_i)x_i}=0$, mas aun $\sum{a_i-b_i}=\sum{a_i}-\sum{b_i}=0$.
    Entonces, vemos que $a_i-b_i=0$ lo que hace  $a_i=b_i$.

    Por ultimo, suponga que $A$ es abaracado por  $x_0, \dotms x_m$ y que todo
    $x \in A$ se puede escribir unicamente como una combinacion af\'in de  $x_0,
    \dots, x_m$. Es decir, $x=\sum{a_ix_i}$. S\'i $m=1$, tenemos el resultado.
    Ahora, suponga quie  $m \geq 1$. Suponga que  $x_1-x_0, \dotsm x_m-x_0$ son
    linealmente dependientes. Entonces existen $r_i \in \R$, no todos  $0$ con
    $\sum{r_i(x_i-x_0)}=0$. Entonces para una $r_j \neq 0$.
    \begin{equation*}
        \sum{\frac{r_i}{r_j}(x_i-x_0)}=0
    \end{equation*}
    Suponga, sin perder la generalidad, que hay un
    $r_j=1$. Entonces  $x_j \in \{x_0, \dots, x_m\}$ y tiene las
    representaciones:
    \begin{align*}
        x_j     &=      x_j=1 \cdot x_j     \\
        x_j     &=      -\sum_{i \neq j}{r_ix_i}+(1+\sum_{i \neq j}{r_j})x_0    \\
    \end{align*}
    Esto contradice que cada $x \in X$ tiene representaci\'on unica, por lo
    tanto  $x_1-x_0, \dots, x_m-x_0$ tienen que ser linealmente indpendiente,
    por lo tanto $x_0, \dots, x_m$ es af\'in independiente.
\end{proof}
\begin{corollary}
    Una combinaci\'on af\'in es independiented del orden en lo que esta dado.
\end{corollary}
\begin{corollary}
    S\'i $A$ es af\'in en  $\R^n$, y esta abarcado por  $p_0, \dots, p_m$,
    af\'in independientes, entonces $A$ es una traslaci\'on de un subespacio
    vectorial $m$-dimensional de $\R^n$. A saber,  $A=V+x_0$.
\end{corollary}
\begin{proof}
    Sean $x_0=p_0$ y $V$ el subespacio vectorial de $\R^n$ con base  $\{p_1-p_0,
    \dots, p_m-p_0\}$, Por un lado, s\'i $z \in A$, entonces  $z=\sum{t_ip_i}$
    donde $\sum{t_i}=1$. Entonces
    $z=\sum{t_ip_i}+t_0p_0=\sum{t_ip_i}-\sum{t_ip_0}+(t_0+\sum{t_i})p_0=
    \sum{t_i(p_i-p_0)}+p_0 \in V+x_0$. Por otro lado, s\'i $z \in V+x_0$,
    entonces $z=\sum{t_i(p_i-p_0)}+x_0=\sum{t_i(p_i-p_0)}+p_0=\sum{t_ip_i}$ y
    vemos que $\sum{t_i}=1$, as\'i que $z \in Z$.
\end{proof}

\begin{definition}
    Sea $x_0, \dots, x_m \in \R^n$ af\'in indpendientes. S\'i $t_0, \dots, t_m
    \in \R$ tales que $x=\sum{t_ix_i}$, llamamos a los $t_ia_i$ las
    \textbf{coordinadas baricentricas} de $x \in \R^n$ y lo representamos como
    $(t_0x_0, \dots, y_m,x_m)$.
\end{definition}

\begin{definition}
    Sea $x_0, \dots, x_m \in \R^n$ af\'in independiente. Llamamos al casco
    convexo, $[x_0, \dots, x_m]$ abarcado por los puntos un
    \textbf{$m$-simplejo} con \textbf{v\'ertices} $x_0, \dots, x_m$. Llamamos el
    $m$-simplejo  $[e_0, \dots, e_m]$ el $m$-simplejo \textbf{estandar}, donde
    $\{e_0, \dots, e_m\}$ es el base estandar de $\R^{m+1}$, y lo denotamos
    $\Delta^{m}$.
\end{definition}

\begin{lemma}\label{10.27}
    Sea $[e_0, \dots, e_m]$ el $m$-simplejo estandar. Entonces las coordinadas
    baricentricas de un punto $x \in [e_0, \dots, e_m]$ coenciden con las
    coordinadas cartesianas de $x$.
\end{lemma}

\begin{definition}
    Sea $[x_0, \dots, x_m]$ un $m$-simplejo. La \textbf{cara opuesta} a $x_i$,
    para  $0 \leq i \leq m$ es el $m-1$-simplejo, denotado  $[x_0, \dots, \hat{x_i},
    \dots, x_m]=\{\sum{a_jx_j} : a_j \geq 0, \sum{a_j}=1, a_i=0\}$. Para $0 \leq
    k \leq m-1$, una \textbf{$k$-cara} es un $k$-simplejo formado por $k+1$
    elementos de un $m$-simplejo.
\end{definition}

\begin{definition}
    Definimos el \textbf{borde} de un $m$-simplejo  $[x_0, \dots, x_m]$ de ser
    la union de todas las caras opuestas a $x_i$, para  $0 \leq i \leq m$. Lo
    denotamos como  $\partial{[x_0, \dots, x_m]}$.
\end{definition}

\begin{example}\label{}
    Las $1$-caras del  $3$-simplejo  $[x_0,x_1,x_2,x_3]$ son $[x_0,x_1],
    [x_0,x_3],[x_1,x_2], [x_1,x_3]$, $[x_2,x_3]$, y $[x_0,x_3]$. Nota que hay
    ${4 \choose 2}=6$ $1$-caras. Nota que los $1$-caras son los aristas del
    tetrahedo.
\end{example}

\begin{definition}
    Definimos al \textbf{baricentro} de un $m$-simplejo $[x_0, \dots, x_m]$ de
    ser el punto $\frac{1}{m+1}\sum{x_i}$
\end{definition}

\begin{theorem}\label{10.28}
    Denote al $n$-simplejo  $[x_0, \dots, x_n]$ por $S$. Entonces:
    \begin{enumerate}
        \item[(1)] S\'i $u,v \in S$, entonces $\|u-v\| \leq \sup_i{\|u-x_i\|}$.

        \item[(2)] $\diam{S}=\sup_{i,j}{\|x_i-x_j\|}$.

        \item[(3)] S\'i $b$ es el baricentro de  $S$, entonces  $\|b-x_i\| \leq
            \frac{n}{n+1}\diam{S}$
\end{theorem}
\begin{proof}
    Sean $u,v \in S$ y  $v=\sum{t_ix_i}$ donde $\sum{t_i}=1$ y $t_i \geq 0$.
    Entonces $\|u-v\|=\|u\sum{t_i}=\sum{t_iv_i}\| \leq \sumd{t_i\|u-x_i\|} \leq
    \sum{t_i}\sup_i{\|i-x_i\|}=\sup_i{u-x_i}$.

    Ahora, note que por la condicion de arriba, $\|u-x_i\| \leq
    \sup_j{\|x_i-x_j\|} \leq
    \sup_i{(\sup_j{\|x_i-x_j\|})}=\sup_{i,j}{\|x_i-x_j\|}$. Por lo tanto,
    $\diam{S} \leq \sup_{i,j}{\|x_i-x_j\|}$.

    Por ultimo, sea $b=\frac{1}{n+1}{\sum{x_i}}$. Entonces tenemos que
    $\|b-x_i\|=\|\sum{\frac{1}{n+1}{x_j}}-p_i\|=\|\sum{\frac{1}{n+1}x_j}-
    \sum{\frac{1}{n+1}}x_i\| \leq \frac{1}{n+1}\sum{\|x_j-x_i\|} \leq
    \frac{1}{n+1}\sum{\sup_{i,j}{\|x_j-x_i\|}} \leq
    \frac{n}{n+1}\sup{\|x_j-x_i\|}$.
\end{proof}

\begin{definition}
    Sea $x_0, \dots, x_m \in \R^n$ af\'in independientes y sea $A$ el conjunto
    af\'in abarcado por estos puntos. Una  \textbf{aplicaci\'on af\'in} es una
    mapa $T:A \xrightarrow{} \R^k$, para $1 \leq k \leq n$ tal que:
    \begin{equation*}
        T(\sum{t_ix_i})=\sum{t_it(x_i)}
    \end{equation*}
    donde $\sum{t_i}=1$.
\end{definition}

\begin{lemma}\label{10.29}
    Una aplicaci\'on af\'in preserva las combinaciones convexos.
\end{lemma}

\begin{theorem}\label{10.30}
    Sean $[x_0, \dots, x_m]$ y $[y_0, \dots, y_n]$ $m$ y $n$-simplejos,
    respectivamente. S\'i $f:\{x_0, \dots, x_m\} \xrightarrow{} [y_0, \dots,
    y_n]$ es una mapa, entonces existe una unica aplicaion afin $T:[x_0, \dots,
    x_m] \xrightarrow{} [y_0, \dots, y_n]$ tal que $T(x_i)=f(x_i)$ para todo $1
    \leq i \leq m$.
\end{theorem}
\begin{proof}
    Defina $T(\sum{t_ix_i})=\sum{t_if(x_i)}$, donde $\sum{t_i}=1$. Esta mapa es
    una aplicacion afin, y su unicidad es consequencia de la unicidad de las
    coordenadas baricentricas.
\end{proof}
