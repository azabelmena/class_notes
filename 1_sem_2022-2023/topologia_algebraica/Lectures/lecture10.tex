\section*{Lectura 10: Simplejas.}

\begin{definition}
    Un subconjunto $A \subseteq \R^n$ es  \textbf{af\'in} s\'i para todo $x,y
    \in A$, la recta  $l$ que pasa por  $x$ y  $y$ esta contenido en  $A$.
\end{definition}

\begin{lemma}\label{10.23}
    S\'i $A \subseteq \R^n$ es af\'in, entonces es convexo.
\end{lemma}

\begin{example}\label{}
    Los conjuntos $\emptyset$ y $\{x\}$, con $x \in \R^n$ son af\'in.
\end{example}

\begin{theorem}\label{10.24}
    S\'i $\{X\alpha\}$ es una colecci\'on de conjuntos afines en $\R^n$,
    entonces el interseccion de todos es afin.
\end{theorem}
\begin{proof}
    Sea $X=\bigcap{X_\alpha}$ con $X_\alpha \in \R^n$ af\'in. Sean  $x,y \in X$
    y  $l(x,y)$ la recta que pasa por $x$ y  $y$. Entonces  $l(x,y) \in
    X_\alpha$ para todo $\alpha$ lo cual hace  $l(x,y) \in X$. Por lo tanto $X$
    es af\'in.
\end{proof}
\begin{corollary}
    S\'i $\{X\alpha\}$ es una colecci\'on de conjuntos convexos en $\R^n$,
    entonces el interseccion de todos es convexos.
\end{corollary}

\begin{definition}
    Sea $X \subseteq \R^n$. Llamamos al interseccion de todos los conjuntos
    afines conteniendo a $X$ el  \textbf{casco af\'in} de $X$. De igual forma,
    la interseccion de todos conjuntos convexos conteniendo a  $X$ se llama el
     \textbf{casco convex} generado $X$. En ambos casos, denotamos el casco afin
     o convexo como $[X]$.
\end{definition}

\begin{definition}
    Una \textbf{combianci\'on af\'in} de puntos $x_0, \dots, x_m \in \R^n$ es un
    punto $x$ tal que:
    \begin{equation*}
        x=t_0x_0+\dots+t_mx_m
    \end{equation*}
    donde $\sum{t_i}=1$. Una \textbf{combinaci\'on convexo} es una combinacion
    afin donde $y_i \geq 0$ para todo  $1 \leq i \leq m$.
\end{definition}
