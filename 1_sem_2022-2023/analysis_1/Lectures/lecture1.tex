\section*{Lecture 1: Review}

We begin with a review of some prelimanary results of set theory and advanced
calculus.

\begin{definition}
    Let $A$ and $B$ be sets, and let $f:A \rightarrow B$ be a map of $A$ into
   $B$. We say that $f$ is  \textbf{1--1} if for every  $x,y \in X$,  $f(x)=f(y)$
   implies $x=y$. We say $f$ is \textbf{onto} if for every $y \in B$, there is
   an  $x \in A$ for which $y=f(x)$. We say that $f$ is a  \textbf{1--1
   correspondence} of $A$  \textbf{onto} $B$ if  $f$ is both 1--1 and onto and
   onto.
\end{definition}

\begin{theorem}[Beinstern's Theorem]\label{thm_1}
    Let $A$ and  $B$ be sets. If there is a 1--1 map of $A$ into  $B$, and a
    1--1 map of  $B$ into $A$, then there exists a 1--1 correspondence of
    $A$ onto  $B$.
\end{theorem}

\begin{axiom}[The Axiom of Choice]\label{axm_1}
    Suppose that $\{A_\alpha\}_{\alpha \in \Lambda}$ is a collection of nonmepty
    sets indexed by the set $\Lambda$. Then there exists a map $f:\Lambda
    \rightarrow \bigcup{A_\alpha}$ called a \textbf{choice function}, defined by
    the rule $f(\alpha) \in A_\alpha$ for all $\alpha \in \Lambda$.
\end{axiom}
\begin{remark}
    What this axiom says is that given any (not necesarrily countable)
    collection of sets, one can ``choose '' an element from each set.
\end{remark}

\begin{definition}
    We define an \textbf{order} on a set $A$ to be a relation  $<$ satisfying
    the following properties for all $a,b,c \in A$:.
    \begin{enumerate}
        \item[(1)] $a<a$  (Reflexive).

        \item[(2)] If $a<b$ and  $b<a$, then  $a=b$  (Antisymmetry).

        \item[(3)] If $a<b$ and  $b<c$, then  $a<c$  (Transitivity).
    \end{enumerate}
     We say that elements $a,b \in A$ are \textbf{comparable} under $<$ if
     either $a<b$ or $b<a$. If every element of $A$ is comparable, then  $<$ is
     called a  \textbf{total order}.
\end{definition}

\begin{definition}
    Let $A$ be a set with order  $<$. We call an element  $x \in A$ a
    \textbf{maximum} of $A$ if  $x<a$ implies  $x=a$ for all $a \in A$. If $B
    \subseteq A$, then we call an element $a \in A$ an \textbf{upper bound} of
    $B$ if  $b<a$ for all  $B$ in  $B$; and we say  $B$ is  \textbf{bounded
    above}. If $b<a'$ implies that  $a<a'$ for any  $a' \in A$, then we call
    $a$ a  \textbf{least upper bound} of $B$ and write  $\sup{B}=a$.
\end{definition}

\begin{definition}
    Let $A$ be a set with order  $<$. We call an element  $x \in A$ a
    \textbf{minimum} of $A$ if  $a<x$ implies  $x=a$ for all $a \in A$. If $B
    \subseteq A$, then we call an element $a \in A$ an \textbf{lower bound} of
    $B$ if  $a<b$ for all  $B$ in  $B$; and we say  $B$ is  \textbf{bounded
    below}. If $a'<b$ implies that  $a'<a$ for any  $a' \in A$, then we call
    $a$ a  \textbf{greatest lower bound} of $B$ and write  $\inf{B}=a$.
\end{definition}

\begin{theorem}[Zorn's Lemma]\label{thm_2}
    Let $A$ be a set with order  $<$. If every totally ordered subset of  $A$
    under  $<$ has an upperbound, then  $A$ has a maximum element.
\end{theorem}
\begin{remark}
    It can be shown that Zorn's lemma and the axiom of choice are equivalent
    statements. That is you can prove Zorn's lemma from the axiom of choice, and
    you can prove the axiom of choice from Zorn's lemma.
\end{remark}

\begin{theorem}\label{thm_3}
    For any sets $A$ and  $B$, there is either a 1--1 map of  $A$ into  $B$, or
    a 1--1 map of  $B$ into  $A$.
\end{theorem}
\begin{proof}
    Consider the collection of all triples $C=\{(X,Y,f) : f:X \rightarrow Y
    \text{ is 1--1 and onto, where }\\ X \subseteq A, Y \subseteq B\}$, and
    define an order $<$ on  $C$ by: $(X,Y,f)<(X',Y',g)$ if, and only if $X
    \subseteq X'$,  $Y \subseteq Y'$, and  $g|_X=f$. Now suppose that the
    collection  $\{(X_\alpha, Y_\alpha,f_\alpha)\}$ is a totally ordered subset
    of $C$. Then define
    \begin{align*}
        \tilde{X}   &=      \bigcup_{\alpha}{X_\alpha} \\
        \tilde{Y}   &=      \bigcup_{\alpha}{Y_\alpha} \\
    \end{align*}
    and since for all $x \in \tilde{X}$, there is an $\alpha$  for which $x \in
    X_\alpha$, define the map $\tilde{f}:\tilde{X} \rightarrow \tilde{Y}$ by the
    rule $\tilde{f}(x)=f_\alpha(x) \in Y_\alpha$. This map is well defined by
    the total order on $\{(X_\alpha,Y_\alpha,f_\alpha)\}$.

    Now, notice that $\tilde{f}$ is an upperbound of the collection
    $\{(X_\alpha,Y_\alpha_f_\alpha)\}$, therefore, by Zorn's lemma there is a
    maximum element $(X,Y,f)$ of $C$. By definition we get that $f$ is a 1--1
    correspondence of $X$ onto $Y$, and by maximality, we get that either $X=A$
    or $Y=B$. Suppose on the contary that this is not true. Then there is an  $a
    \in \com{A}{X}$ and a $b \n \com{B}{Y}$. Letting $X'=X \cup a$ and  $Y'=Y
    \cup b$, and  $f':X' \rightarrow Y'$ defined by $f'|_X=f$ and  $f'(a)=b$.
    Then $(X,Y,f)<(X',Y',f')$, which contradicts the maximality of $(X,A,f)$.

    Therefore, we must have that either $X=A$ or  $Y=B$. Now, if  $X=A$, then
    $f:A \rightarrow Y$ is 1--1 and onto, taking the extension of $Y$ to  $B$,
    $f_B:A \rightarrow B$, we see it must be 1--1 as well. By the same reasoning
    we can assure there is a 1--1 map $f_A:B \rightarrow A$ of $B$ into  $A$.
\end{proof}
