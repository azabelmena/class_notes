\section*{Lecture 1: Semi-Algebras, Algebras, and $\sigma$-Algebras}

\begin{definition}
    Let $R$ be a set, we call a set  $S \subseteq 2^R$ a  \textbf{semi-algebra}
    if
    \begin{enumerate}
        \item[(1)] $R \in S$.

        \item[(2)] If $A,B \in S$, then  $A \cap B \in S$.

        \item[(3)] If $A \in S$, then there is a finite collection of disjoints
            elements $\{E_i\}_{i=1}^n$ for which
            \begin{equation*}
                \com{S}{A}=\bigcup_{i=1}^n{E_i}
            \end{equation*}
    \end{enumerate}
\end{definition}

\begin{example}\label{example_1}
    Consider $\R$ the real numbers, and let $S$ be the collection
    \begin{equation*}
        S=\{(a,b] : a<b, \text{ and } a,b \in \R\} \cup \{(-\infty,b]\}_{b \in
        \R} \cup \{(a,\infty)\}_{a \in \R} \cup \emptyset
    \end{equation*}
    Then $S$ is a semi-algebra on $\R$.
\end{example}

\begin{definition}
    Let $R$ be a set, and let  $S \subseteq 2^\R$. We call $S$ an
    \textbf{algebra} if
    \begin{enumerate}
        \item[(1)] $R \in S$.

        \item[(2)] If $A,B \in S$, then  $A \cap B \in S$.

        \item[(3)] If $A \in S$, then  $\com{S}{A} \in S$.
    \end{enumerate}
\end{definition}

\begin{definition}
    Let $R$ be a set, and let  $S \subseteq 2^\R$. We call $S$ an
    \textbf{$\sigma$-algebra} if
    \begin{enumerate}
        \item[(1)] $R \in S$.

        \item[(2)] If $\{A_j\}$ is a countable collection of elements of $S$,
            then $\bigcap{A_n} \in S$.

        \item[(3)] If $A \in S$, then  $\com{S}{A} \in S$.
    \end{enumerate}
\end{definition}

\begin{lemma}\label{lemma_1}
    For any set $R$, if  $S$ is a $\sigma$-algebra, then it is an algebra.
\end{lemma}

\begin{lemma}\label{lemma_2}
    For any set $R$, if  $S$ is an algebra, then it is closed under finite
    unions.
\end{lemma}
\begin{proof}
    Notice that $A \cup B=\com{S}{(\com{S}{A} \cap \com{S}{B})}$.
\end{proof}
\begin{corollary}
    algebras are closed under arbitrary unions.
\end{corollary}

\begin{lemma}\label{lemma_3}
    Let $R$ be a set and $\{S_\alpha\}$ a collection (not necessarily countable)
    of $\sigma$-algebras of $R$. Then the set
    \begin{equation*}
        $S=\bigcap{S_\alpha}$
    \end{equation*}
    is a $\sigma$-algebra.
\end{lemma}
\begin{proof}
    Since $R \in S_\alpha$ for all  $\alpha$,  $R \in S$, moreover if  $A,B \in
    S$, then  $A,B \in S_\alpha$ for all  $\alpha$, so that  $A \cap B \in
    \alpha$, this makes $A \cap B \in S$.

    Lastly, let  $A \in S$, then  $A \in S_\alpha$ for all $\alpha$, so that
    $\com{S_\alpha}{A} \in S_\alpha$, Since $\sigma$-algebras are closed under
    arbitrary unions, we get
    $\bigcup{(\com{S_\alpha}{A})}=\com{(\bigcap{S_\alpha})}{A}=\com{S}{A} \in S$.
\end{proof}

\begin{definition}
    Let $R$ be a set, and  $\Cc \subseteq 2^R$. We call the algebra $S$ the
    algebra \textbf{generated} by $\Cc$ if  $\Cc \subseteq S$ and if $\B$ is an
    algebra containing $\Cc$, then  $S \subseteq B$. That is,  $S$ is the
    smallest algebra that contains  $\Cc$.
\end{definition}

\begin{lemma}\label{lemma_4}
    Let $\{S_\alpha\}$ a collection of algebras on a set $R$, containing $\Cc
    \subseteq 2^R$. Then $S=\bigcap{S_\alpha}$ is the algebra generated by
    $\Cc$.
\end{lemma}
\begin{proof}
    We have that $S$ is an algebra by a similar proof to that of lemma
    \ref{lemma_3}, Moreover, by definition, $\Cc \in S$. Now, let  $B$ be an
    algebra, containing  $\Cc$. Then $B$ is in the collection $\{S_\alpha\}$, so
    that $S \subseteq B$.
\end{proof}

\begin{definition}
    Let $R$ be a set, and  $\Cc \subseteq 2^R$. We call the $\sigma$-algebra $S$
    the $\sigma-$algebra \textbf{generated} by $\Cc$ if  $\Cc \subseteq S$ and
    if $\B$ is an $\sigma$-algebra containing $\Cc$, then  $S \subseteq B$. That
    is,  $S$ is the smallest $\sigma$-algebra that contains  $\Cc$.
\end{definition}

\begin{lemma}\label{lemma_5}
    Let $R$ be a set and $S$ a semi-algebra of $R$. Let  $T(S)$ be the algebra
    generated by $S$. Then $A \in T(S)$ if, and only if there exists a finite
    collection $\{E_i\}_{i=1}^n$ of disjoint elements of $S$ for which
    \begin{equation*}
        A=\bigcup_{i=1}^n{E_i}
    \end{equation*}
\end{lemma}
\begin{proof}
    Observe that if $A=\bigcup{E_i}$, since $E_i \in S$ for all  $1 \leq i \leq
    n$, then $E_i \in T(S)$. This makes $A \in T(S)$.

    Conversely, suppose that $A \in T(S)$. Consider the collection
    \begin{equation*}
        \Bc=\{\bigcup{F_j} : F_j \in S \text{ and } F_i \cap F_j=\emptyset
        \text{ whenever } i \neq j\}
    \end{equation*}
    Notce that $S \subseteq \Bc$, and hence, so is $R$. Now, let
    $A=\bigcup{F_i}$ and $B=\bigcup{F_j}$. Then $A \cap B=(\bigcup{F_j}) \cap
    (\bigcup{F_i})=\bigcup{(F_i \cap F_j)}$, which is a disjoint union of
    elements in $S$. So  $A \cap B \in \Bc$ whenever  $A,B \in \Bc$.

    Lastly, let  $A=\bigcap{F_j}$, then
    $\com{\Bc}{A}=\bigcap{(\com{\Bc}{F_j})}=\bigcap{\bigcap_{k_i}^{l_i}{F_{i,k_i}}}$.
    Since $F_{i,k_i} \in S$, we get $\com{\Bc}{A} \in \Bc$, so that $\Bc$ is an
    algebra. This makes $A=T(S) \subseteq \Bc$.
\end{proof}
