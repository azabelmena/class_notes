\section*{Lecture 5: Outer Measure}

We wish to define a ``measure'' on all ``measurable'' sets. That is define a map
$\mu:2^\R \xrightarrow{} [0,\infty)$ such that
\begin{enumerate}
    \item[(1)] $\mu([a,b])=\mu((a,b))=b-a$.

    \item[(2)] $\mu$ is translation invariant, that is for any  $x \in \R$,
        $\mu(A+x)=\mu(A)$ for any $A \subseteq \R$.

    \item[(3)] $\mu$ is countably additive. That is if  $A$ is a subset of
        $\R$, and $\{A_n\}$ is a countable collection of subsets of $A$ that
        cover $A$, then
        \begin{equation*}
            \mu(A)=\sum{\mu(A_n)}
        \end{equation*}
\end{enumerate}

The problem is that no such general map $\mu$ exists. We make the following
definition then.

\begin{definition}
    Let $A$ be a subset of  $\R$, and consider all countable collections of
    intervals $\{I_n\}$ that cover $A$. We define the  \textbf{outer measure} to
    be a map $m^*:2^\R \xrightarrow{} [0,\infty)$ such that
    \begin{equation}
        m^*(A)=\inf{\{\sum{l(I_n)}\}}
    \end{equation}
    for all $n$, where $l(I_n)$ is the \textbf{length} of $I_n$ which is defined
    to be the difference of the endpoints if $I_n$ is bounded, and $\infty$
    otherwise.
\end{definition}

\begin{lemma}\label{lemma_4.37}
    $\m^*(\emptyset)=0$ and if $A$ and  $B$ are subsets of  $\R$ such that  $A
    \subseteq B$, then  $m^*(A) \leq m^*(B)$.
\end{lemma}

\begin{theorem}\label{thm_4.38}
    The outer measure satisfies:
    \begin{enumerate}
        \item[(1)] $m*([a,b])=m^*((a,b))=b-a$.

        \item[(2)] $m^*$ is translation invariant

        \item[(3)] $m^*$ is countably subadditive. That is  if  $A$ is a subset
            of $\R$, and $\{A_n\}$ is a countable collection of subsets of $A$
            that cover $A$, then
            \begin{equation*}
                \mu(A) \leq \sum{\mu(A_n)}
            \end{equation*}
    \end{enumerate}
\end{theorem}

\begin{definition}
    A set $A$ is  \textbf{measurable} if for all $E \subseteq \R$ :
    \begin{equation}
        m^*(E)=m^*(E \cap A)+m^*(\com{E}{A})
    \end{equation}
    That is for any $E$ that $A$ cuts into two subsets, then the outer measure
    of $E$ is additive for those two subsets.
\end{definition}
\begin{remark}
    Note that it is possible that $A$ and  $E$ are disjoint, so that  $m^*(E
    \cap A)=0$ and $m^*(\com{E}{A})=m^*(A)$.
\end{remark}

\begin{lemma}\label{lemma.39}
    If $E$ is countable, then $E$ has outer measure $0$. Moreover, if  $E$ has
    outer measure  $0$, then  $E$ is measurable.
\end{lemma}

\begin{lemma}\label{leamm_4.40}
    The union of two measurable sets is measurable.
\end{lemma}

\begin{HW}
    Problem $55$ on pg. $28$ (Royden \& Fitzpatrick).
\end{HW}
