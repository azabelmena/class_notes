\section*{Lecture 3}

\begin{theorem}\label{thm_3.11}
    If $A \subseteq \R$ is a nonempty set bounded below, then $A$ has a greatest
    lowerbound in $\R$.
\end{theorem}

\begin{lemma}\label{lemma_3.12}
    For every $A$ nonempty,  $\inf{A} \leq \sup{A}$
\end{lemma}

\begin{example}\label{}
    For $A=\emptyset$, $\inf{A}=\infty$ and $\sup{A}=-\infty$.
\end{example}

\begin{definition}
    We define the \textbf{extended real numbers} to be $\R$ together with the
    symbols  $-\infty$ and  $\infty$.
\end{definition}

\begin{theorem}\label{thm_3.13}
    The following are true for the extended real numbers:
    \begin{enumerate}
        \item[(1)] For any $a \in \R$,  $a+\infty=\infty$ and
            $a-\infty=-\infty$.

        \item[(2)] $\infty+\infty=\infty$.

        \item[(3)] $\infty \cdot \infty=\infty$ and  $\infty \cdot
            (-\infty)=\infty$.
    \end{enumerate}
\end{theorem}

\begin{theorem}\label{thm_3.14}
    The closed interval $[a,b]$, for $a,b \in \R$ is uncountable.
\end{theorem}
\begin{proof}
    Suppose that $[a,b]$ is countable. Then list the elements of $[a,b]$ as:
    \begin{align*}
        x_0=a   &&  x_1     &&      \dots       &&      x_n     &&  \dots   \\
    \end{align*}
    Now choose an $a_1,a_2 \in [a,b]$ such that $[a_1,b_1] \subseteq [a,b]$ and
    $x_1 \notin [a_1,b_1]$. Choose then $a_2,b_2 \in [a_1,b_1]$ for which $x_2
    \notin [a_2,b_2] \subseteq [a_1,b_1]$. Porceeding inductively, then
    construct the sequence of intervals $\{a_n,b_n\}$ where $[a_{n+1},b_{n+1}]
    \susbeteq [a_n,b_n]$ and $x_n \notin [a_n,b_n]$. Notice that the sequence of
    endpoints $\{a_n\}$ has a least upperbound  $a'=\sup{\{a_n\}}$ such that
    $a_n<a'<b_n$ for all  $n \in \N$. Therefore  $a' \in [a_n,b_n]$ so that $a'
    \neq x_n$ for any  $n$. This contradicts the complete listing of  $[a,b]$ as
    there is an element $a'$ that cannot be listed.
\end{proof}
\begin{corollary}
    $\R$ is uncountable.
\end{corollary}
\begin{proof}
    Notice that $\R$ is homeomorphic to the closed interval  $[0,1]$.
\end{proof}
\begin{corollary}
    $|2^\N|=|[0,1]|$.
\end{corollary}

\begin{definition}
    We call a set $A \subseteq \R$  \textbf{open} in $\R$ if there exists an
    interval  $(a,b)$ such that $x \in (a,b) \subseteq A$, for $a,b \in \R$. We
    call  $A$  \textbf{closed} in $\R$ if  $\com{\R}{A}$ is open in $\R$.
\end{definition}

\begin{example}\label{}
    \begin{enumerate}
        \item[(1)] $\R$ is open in itself, since  $\R=(-\infty,\infty)$.

        \item[(2)] Every open interval $(a,b)$ of $\R$ is open in $\R$ since
            they contain themselves.
    \end{enumerate}
\end{example}

\begin{lemma}\label{lemma_3.15}
    $\Q$ is dense in  $\R$.
\end{lemma}
\begin{proof}
\end{proof}

\begin{theorem}\label{thm_3.16}
    If $A$ is open in  $\R$, the  $A$ is the countable union of disjoint open
    intervals.
\end{theorem}
\begin{proof}
    Let $A$ be open in  $\R$. Then for each  $x \in A$, there exists  $a,b \in
    \R$ for which  $x \in (a,b) \subseteq A$. Now consider the sets $Z=\{z:(x,z)
    \subseteq A\}$ and $W=\{w:(w,x) \subseteq A\}$. We have $Z$ and  $W$ are
    nonempty since  $b \in Z$ and  $a \in W$, and they are bounded above and
    below by  $b$ and  $a$, respectively. Now, let $b_x=\sup{Z}$ and
    $a_x=\inf{W}$. Then we have that $x \in (a_x,b_x) \subseteq A$, so that
    $a_x<x<b_x$.

    Moreover, let $x \in (a_x,b_x)$ such that $x<c$, without loss of generality.
    Then $c<b_x$ which means that $c$ is not an upperbound of $x$; i.e. there is
    a $z \in Z$ such that  $(x,z) \subseteq A$ with $z>c$.

    Suppose that $b_X \in A$, then there is a  $\delta>0$ such that
    $(b_x-\delta, b_x+\delta) \subseteq A$, so $(a_x,b_x) \cup
    (b_x-\delta,b_x+\delta)=(a_x,b_x+\delta) \subseteq A$, which contradicts
    that $b_x=\sup{Z}$. So $b_x \notin A$. Similarly, we have that $a_X \notin
    A$. This shows that $(a_x,b_x)$ is maximally chosen.

    Now, consider the collection $\{(a_x,b_x)\}_{x \in A}$ of all intervals
    contained in $A$, containing $x$ for all $x \in A$,  with $(a_x,b_x) \cap
    (a_y,b_y)=\emptyset$ whenever $x \neq y$. Let
    \begin{equation*}
        I=\bigcup_{x \in X}{(a_x,b_x)}
    \end{equation*}
    Then $A=I$. Moreover, by the density of $\Q$ in  $\R$, each  $(a_x,b_x)$
    contains a rational $q$, so taking the map $\{(a_x,b_x)\} \xrightarrow{}
    \Q$, by the rule $(a_x,b_x) \xrightarrow{} q$ if $q \in (a_x,b_x)$, we see
    that this map is onto. Therefore $\{(a_x,b_x)\}$ is a countable collection.
    Therefore $A$ is the union of countably many disjoint open intervals.
\end{proof}

\begin{theorem}[Heine-Borel]\label{thm_3.17}
    If $F \subseteq \R$ is bounded, closed, and nonempty, then  $F$ is compact
    in  $\R$.
\end{theorem}

\begin{theorem}[Nests of Closed Sets]\label{thm_3.18}
    Suppose that $\{F_n\}$ is a collection of closed, bounded, and nonempty sets
    such that $F_{n+1} \subseteq F_n$ for all $n \in \N$.  Then  $\bigcap{F_n}$
    is nonempty.
\end{theorem}

\begin{definition}
    Let $X$ be a set and  $\Ac$ a collection of subsets of  $X$. We call  $\Ac$
    a  \textbf{$\sigma$-algebra} if:
    \begin{enumerate}
        \item[(1)] $\emptyset, X \in \Ac$.

        \item[(2)] $\Ac$ is closed under countable unions of subsets of  $X$.

        \item[(3)] If $Y \in \Ac$, then  $\com{X}{Y} \in \Ac$.
    \end{enumerate}
\end{definition}
\begin{theorem}\label{thm_3.19}
     For any collection $\{A_\alpha\}$ of subsets of a set $X$, there exists a
     smallest  $\sigma$-algebra containing  $\{A_\alpha\}$.
\end{theorem}

\begin{definition}
    We call the smallest $\sigma$-algebra containing all open intervals in  $\R$
    the  \textbf{Borel algebra} and denotes it $\Bc$. Elements of  $\Bc$ are
    called  \textbf{Borel sets}.
\end{definition}

\begin{lemma}\label{lemma_3.20}
    $\Bc$ contains all closed sets in  $\R$.
\end{lemma}

\begin{lemma}\label{lemma_3.21}
    $\Bc$ contains all countable intersections of open intervals of  $\R$ and
    all countable unions of closed sets in  $\R$.
\end{lemma}
