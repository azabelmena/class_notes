\section*{Lecture 4}

\begin{definition}
    Let $\{A_n\}$ be a collection of sets. We define the \textbf{limit superior}
    of $\{A_n\}$ to be $\lim\sup{\{A_n\}}=\{x \text{ which is in infinitely many
    } A_n\}$. We define the \textbf{limit inferiro} of $\{A_n\}$ to be
    $\lim\inf{\{A_n\}}=\{x \text{ which is in almost all $A_n$}\}$.
\end{definition}

\begin{lemma}\label{lemma_4.22}
    Let $\{A_n\}$ be a collection of sets, then $\lim\inf{\{A_n\}} \subseteq
    \lin\sup{\{A_n\}}$.
\end{lemma}

\begin{lemma}\label{lemma_4.23}
    Let $\{A_n\}$ be a collection of sets. If $x \in \lim\sup{\{A_n\}}$, then
    for $N>0$ arbitrarily large, there exists an  $n \geq N$ such that  $x \in
    A_n$. Likewise, if  $x \in \lim\inf{\{A_n\}}$, then  there exists an $N_0>0$
    such that $x \in A_n$ whenver  $n \geq N_0$.
\end{lemma}

\begin{lemma}\label{lemma_4.24}
    For any collection $\{A_n\}$ of sets, we have
    \begin{equation}
        \lim\sup{\{A_n\}}=\bigcap_{N=1}^\infty{(\bigcup_{n=N}^\infty{A_n})}
    \end{equation}
    and
    \begin{equation}
        \lim\inf{\{A_n\}}=\bigcup_{N=1}^\infty{(\bigcap_{n=N}^\infty{A_n})}
    \end{equation}
\end{lemma}

\begin{definition}
    Let $\{a_n\}$ be a sequence of real numbers. We say that $\{a_n\}$
    \textbf{converges} to an $a \in \R$ if for every  $\epsilon>0$, there exists
    a  $N>0$ such that $|a_n-a|<\epsilon$ whenever $n \geq N$. We write
    $\lim_{n \xrightarrow{} \infty}{\{a_n\}}=a$, or $\{a_n\} \xrightarrow{} a$
    as $n \xrightarrow{} \infty$; or we simply omit the $n \xrightarrow{}
    \infty$ with clear enough context.
\end{definition}

\begin{lemma}\label{lemma_4.25}
    If the limit of a sequence exists, then it is unique. Moreover, convergent
    sequneces are bounded, and if $\{a_n\}$ is such a sequence such that $a_n
    \leq c$ for all  $n$, then  $a \leq c$, where  $c \in \R$.
\end{lemma}

\begin{lemma}\label{lemma_4.26}
    If $\{a_n\}$ is a monotone increaseing sequence bounded above, then
    $\{a_n\}$ converges.
\end{lemma}

\begin{theorem}\label{thm_4.27}
    Every bounded sequence has a convergent subsequence.
\end{theorem}

\begin{definition}
    We call a sequence $\{a_n\}$ a \textbf{Cauchy sequence} (or simply
    \textbf{Cauchy}) if for every $\epsilon>0$, there is an  $N>0$ such that
    $|a_n-a_m|/<\epsilon$ whenever  $n,m \geq N$.
\end{definition}

\begin{theorem}\label{thm_4.28}
    A sequnece in $\R$ converges if, and only if it is Cauchy.
\end{theorem}

\begin{theorem}\label{thm_4.29}
    Let $\{a_n\}$ and $\{B_n\}$ be real sequences such that $\lim{\{a_n\}}=a$
    and $\lim{\{b_n\}}=b$. Then the following are true:
    \begin{enumerate}
        \item[(1)] $\lim{\alpha a_n}$ exists and $\lim{\alpha a_n}=\alpha a$,
            for some $\alpha \in \R$.

        \item[(2)] $\lim{\{a_n+b_n\}}$ exists and $\lim{\{a_n+b_n\}}=a+b$.

        \item[(3)] $\lim{\{a_nb_n\}}$ exists and $\lim{\{a_nb_n\}}=ab$.

        \item[(4)] If $a_n \leq b_n$ for all  $n$, then  $a \leq b$.
    \end{enumerate}
\end{theorem}

\begin{definition}
    Let $\{a_n\}$ be a sequence of real numbers. We define the \textbf{limit
    superior} of  $\{a_n\}$ to be $\lim\sup{\{a_n\}}=\lim{(\sup{a_k})}$ as $n
    \xrightarrow{} \infty$ and we define the \textbf{limit inferior} of
    $\{a_n\}$ to be $\lim\inf{\{a_n\}}=\lim{(\inf{\{a_k\}})}$ as $n \xrightarrow{}
    \infty$ for all $k \geq n$.
\end{definition}

\begin{lemma}\label{lemma_4.30}
    For any real sequence $\{a_n\}$:
    \begin{equation*}
        \lim\inf{\{a_n\}} \leq \lim\sup{\{a_n\}}
    \end{equation*}
\end{lemma}
\begin{corollary}
    $\{a_n\}$ converges if and only if $\lim\sup{\{a_n\}}=\lim\inf{\{a_n\}}$.
\end{corollary}

\begin{lemma}\label{lemma_4.31}
    Let $\{a_n\}$ and $\{b_n\}$ be real sequences. Then:
    \begin{enumerate}
        \item[(1)] $\lim\sup{\{a_n\}}=l$ if, and only if there are infinitely
            many $n$ for which  $a_n>l-\epsilon$ and only finitely many  $n$ for
            which $a_n>l+\epsilon$.

        \item[(2)] $\lim\sup{\{a_n\}}=\infty$ if, and only if $\{a_n\}$ is not
            bounded above.

        \item[(3)] $\lim\sup{\{a_n\}}=-\lim\inf{\{-a_n\}}$.

        \item[(4)] If $a_n \leq b_m$ for all  $m,n$, then $\lim\sup{\{a_m\}}
            \leq \lim\inf{\{b_m\}}$.
    \end{enumerate}
\end{lemma}

\begin{definition}
    Let $E \subseteq \R$. We say that a map  $f:E \xrightarrow{} \R$ is
    \textbf{continuous} at a point $a \in E$ if for every  $\epsilon>0$ there
    exists a  $\delta>0$ such that  $|f(x)-f(a)|<\epsilon$ whenever
    $0<|x-a|<\delta$. We say that  $f$ is  \textbf{continuous} on all of $E$ if
    it is continuous at every point of  $E$.
\end{definition}

\begin{theorem}\label{thm_4.32}
    A realvalued function $f:E \xrightarrow{} \R$ is continuous on $E$ if, and
    only if for any $U$ open in $E$, there is a  $V$ open in  $\R$ such that
    $\inv{f}(U)=V \cap E$.
\end{theorem}

\begin{theorem}[The Sequential Criterion]\label{thm_4.33}
    A realvalued function $f:E \xrightarrow{} \R$ is continuous at a point $x'
    \in E$ if, and only if for any sequence of points  $\{x_n\}$ of $E$, with
    $\{x_n\} \xrightarrow{} x'$, we have $\lim{\{f(x_n)\}}=f(x')$.
\end{theorem}

\begin{theorem}[The Extreme Value Theorem]\label{thm_4.34}
    A continuous map on a closed and bounded set takes on a maximum value, or a
    minumum value.
\end{theorem}

\begin{theorem}[The Intermediate Value Theorem]\label{thm_4.35}
    If $f:[a,b] \xrightarrow{} \R$ is continuous on $[a,b]$, then $f([a,b])$ is
    a closed interval in $\R$, with  $f([a,b])=[f(a),f(b)]$ or
    $f([a,b])=[f(b),f(a)]$.
\end{theorem}

\begin{definition}
    A map $f:E \xrightarrow{} \R$ is \textbf{Lipshitz} if for there exists a $c
    \in \R^+$, such that for every $x,y \in E$, we have  $|f(x)-f(y)| \leq
    c|x-y|$.
\end{definition}

\begin{definition}
    We call a map $f:E \xrightarrow{} \R$ \textbf{uniformly continuous} if for
    every $\epsilon>0$, there is a  $\delta>0$ such that if  $x,y \in E$ with
    $|x-y|<\delta$, then  $|f(x)-f(y)|<\epsilon$.
\end{definition}

\begin{theorem}\label{thm_4.36}
    If $f:E \xrightarrow{} \R$ is a continuous map on a closed and bounded set
    $E$, then  $f$ is uniformly continuous.
\end{theorem}
