\section*{Lectura 8: Las Teoremas de Sylow}

\begin{definition}
    Sea $p \in \Z^+$ un primo. Llamos a un grupo $G$ un \textbf{$p$-grupo} s\'i
    cada $g \in G$ es una potencia de  $p$.
\end{definition}

\begin{example}\label{}
    \begin{enumerate}
        \item[(1)] El grupo Klein $V_4=\faktor{\Z}{2\Z} \oplus \faktor{\Z}{2\Z}$
            es un $2$-grupo.

        \item[(2)] Los grupose $\faktor{\Z}{14\Z} \oplus \faktor{\Z}{2\Z}$ y
            $D_{16}$ son $2$-grupos.

        \item[(3)] El grupo $\bigoplus_{n=1}^{\infty}{\faktor{\Z}{5^n\Z}}$ es un
            $5$-grupo, pero  $\prod_{n=1}^{\infty}{\faktor{\Z}{5^n\Z}}$ solo es
            un $5$-grupo cuando $n=1$.
    \end{enumerate}
\end{example}

\begin{definition}
    S\'i $G$ es un grupo con orden  $p^rm$ donde  $p$ es primo y  $p \not| m$,
    entonces llamamos un subgrupo  $P \leq G$ un \textbf{$p$-subgrupo de Sylow},
    o un \textbf{$p$-Sylow} s\'i $\ord{P}=p^r$.
\end{definition}

\begin{lemma}\label{8.31}
    S\'i $G$ es u grupo de orden  $p^rm$ con  $p$ primo, y $p \not|m$ y $P \leq
    G$ es un  $p$-Sylow de  $G$, entonces  $P$ es de orden lo maximo posible.
\end{lemma}
\begin{proof}
    Por el teorema de Lagrange.
\end{proof}

\begin{example}\label{}
    $|D_6|=2^2 \cdot 3$. Nota que $P_1=\{e,r^3,tr^3t\}$,
    $P_2=\{e,r^3,rt,r^4t\}$, y $P_3=\{e,r^3,r^2t,r^5t\}$ son $2$-Sylows de
    $D_6$ y $P=\{e,r,r^4\}$ es $3$-Sylow.
\end{example}

\begin{lemma}\label{8.32}
    S\'i $n=p^rm$ con  $p$ primo y  $p \not| m$, entonces
    \begin{equation*}
        {n \choose p^r} \equiv m \mod{p}
    \end{equation*}
\end{lemma}
\begin{proof}
    Nota que $(x+1)^{p^r}=\sum_{k=1}^{p^r}{{p^r \choose k}x^{p^r-k}}
    \equiv x^{p^rm}+1 \mod{p}$. Entonces $(x+1)^{p^rm} \equiv (x^{p^r}+1)^m
    \mod{p}$, as\'i que
    \begin{equation*}
        \sum{{p^rm \choose k}x^{p^rm-k}} \equiv \sum{{m \choose
        k}(x^{p^r})^{m-k}} \mod{p}
    \end{equation*}
    Mirando el coeficiente de $x^{p^r}$, en la izquierd, tenemos que este
    termino occure cuando $k=p^r(m-1)$, y obtenemeos ${p^rm \choose p^r}={n
    \choose p^r}$. Por el lado derecho, el termino $x^{p^r}$ occure cuando
    $k=m-1$ y por simetria obtenemos ${m \choose 1}=m$.
\end{proof}

\begin{theorem}[El Primer Teorema de Sylow]\label{8.33}
    Sea $G$ un grupo finito de orden  $p^rm$ donde  $p$ es primo, y  $p \not|
    m$. Entonces existe al menos un  $p$-subgrupo de Sylow, de  $G$.
\end{theorem}
\begin{proof}
    Sea $X={G \choose p^r}$. Note que $G$ actua sobre  $X$ v\'a la
    multiplicacion por la izquierda. Ahora, este accion induce en  $X$ una
    particion de  $X$ en orbitas. Es decir
    \begin{equation*}
        {G \choose p^r}=\bigcup{\Oc(S)}
    \end{equation*}
    entonces $p \not| \sum{|\Oc(S)|}$. Por lo tanto, existe un $S \in X$ con $p
    \not| |\Oc(S)|$. Sea $P=\stab{S}$ Entonces por el teoream del
    \'orbita-estabilizador, tenemos
    \begin{equation*}
        |\Oc(S)|=\frac{\ord{G}}{\ord{P}}=\frac{p^rm}{\ord{P}}
    \end{equation*}
    Como $p \not| |\Oc(S)|$, $\ord{P}$ tiene que ser un multiplo de $p^r$, es
    decir que $p^r | \ord{P}$, por lo tanto $p^r \leq \ord{P}$.

    Por otro lado, defina la mapa $\lambda_x:P \xrightarrow{} S$, para $x \in S$
    dado por $\lambda_x:g \xrightarrow{} \lambda_x(g)=g \cdot x$. Vemos que esta
    mapa es bien definida, y que es 1--1. Por lo tanto $\ord{P} \leq |S|=p^r$.
    Por lo tanto $P$ es un  $p$-subugrupo de Sylow.
\end{proof}
