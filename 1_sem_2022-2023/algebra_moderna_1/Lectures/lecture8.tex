\section*{Lectura 8: Las Teoremas de Sylow}

\begin{definition}
    Sea $p \in \Z^+$ un primo. Llamos a un grupo $G$ un \textbf{$p$-grupo} s\'i
    cada $g \in G$ es una potencia de  $p$.
\end{definition}

\begin{example}\label{}
    \begin{enumerate}
        \item[(1)] El grupo Klein $V_4=\faktor{\Z}{2\Z} \oplus \faktor{\Z}{2\Z}$
            es un $2$-grupo.

        \item[(2)] Los grupose $\faktor{\Z}{14\Z} \oplus \faktor{\Z}{2\Z}$ y
            $D_{16}$ son $2$-grupos.

        \item[(3)] El grupo $\bigoplus_{n=1}^{\infty}{\faktor{\Z}{5^n\Z}}$ es un
            $5$-grupo, pero  $\prod_{n=1}^{\infty}{\faktor{\Z}{5^n\Z}}$ solo es
            un $5$-grupo cuando $n=1$.
    \end{enumerate}
\end{example}

\begin{definition}
    S\'i $G$ es un grupo con orden  $p^rm$ donde  $p$ es primo y  $p \not| m$,
    entonces llamamos un subgrupo  $P \leq G$ un \textbf{$p$-subgrupo de Sylow},
    o un \textbf{$p$-Sylow} s\'i $\ord{P}=p^r$.
\end{definition}

\begin{lemma}\label{8.31}
    S\'i $G$ es u grupo de orden  $p^rm$ con  $p$ primo, y $p \not|m$ y $P \leq
    G$ es un  $p$-Sylow de  $G$, entonces  $P$ es de orden lo maximo posible.
\end{lemma}
\begin{proof}
    Por el teorema de Lagrange.
\end{proof}

\begin{example}\label{}
    $|D_6|=2^2 \cdot 3$. Nota que $P_1=\{e,r^3,tr^3t\}$,
    $P_2=\{e,r^3,rt,r^4t\}$, y $P_3=\{e,r^3,r^2t,r^5t\}$ son $2$-Sylows de
    $D_6$ y $P=\{e,r,r^4\}$ es $3$-Sylow.
\end{example}

\begin{lemma}\label{8.32}
    S\'i $n=p^rm$ con  $p$ primo y  $p \not| m$, entonces
    \begin{equation*}
        {n \choose p^r} \equiv m \mod{p}
    \end{equation*}
\end{lemma}
\begin{proof}
    Nota que $(x+1)^{p^r}=\sum_{k=1}^{p^r}{{p^r \choose k}x^{p^r-k}}
    \equiv x^{p^rm}+1 \mod{p}$. Entonces $(x+1)^{p^rm} \equiv (x^{p^r}+1)^m
    \mod{p}$, as\'i que
    \begin{equation*}
        \sum{{p^rm \choose k}x^{p^rm-k}} \equiv \sum{{m \choose
        k}(x^{p^r})^{m-k}} \mod{p}
    \end{equation*}
    Mirando el coeficiente de $x^{p^r}$, en la izquierd, tenemos que este
    termino occure cuando $k=p^r(m-1)$, y obtenemeos ${p^rm \choose p^r}={n
    \choose p^r}$. Por el lado derecho, el termino $x^{p^r}$ occure cuando
    $k=m-1$ y por simetria obtenemos ${m \choose 1}=m$.
\end{proof}

\begin{theorem}[El Primer Teorema de Sylow]\label{8.33}
    Sea $G$ un grupo finito de orden  $p^rm$ donde  $p$ es primo, y  $p \not|
    m$. Entonces existe al menos un  $p$-subgrupo de Sylow, de  $G$.
\end{theorem}
\begin{proof}
    Sea $X={G \choose p^r}$. Note que $G$ actua sobre  $X$ v\'a la
    multiplicacion por la izquierda. Ahora, este accion induce en  $X$ una
    particion de  $X$ en orbitas. Es decir
    \begin{equation*}
        {G \choose p^r}=\bigcup{\Oc(S)}
    \end{equation*}
    entonces $p \not| \sum{|\Oc(S)|}$. Por lo tanto, existe un $S \in X$ con $p
    \not| |\Oc(S)|$. Sea $P=\stab{S}$ Entonces por el teoream del
    \'orbita-estabilizador, tenemos
    \begin{equation*}
        |\Oc(S)|=\frac{\ord{G}}{\ord{P}}=\frac{p^rm}{\ord{P}}
    \end{equation*}
    Como $p \not| |\Oc(S)|$, $\ord{P}$ tiene que ser un multiplo de $p^r$, es
    decir que $p^r | \ord{P}$, por lo tanto $p^r \leq \ord{P}$.

    Por otro lado, defina la mapa $\lambda_x:P \xrightarrow{} S$, para $x \in S$
    dado por $\lambda_x:g \xrightarrow{} \lambda_x(g)=g \cdot x$. Vemos que esta
    mapa es bien definida, y que es 1--1. Por lo tanto $\ord{P} \leq |S|=p^r$.
    Por lo tanto $P$ es un  $p$-subugrupo de Sylow.
\end{proof}

\begin{example}\label{}
    Sea $GL(n,\F_p)$, y escoje una matriz $A \in GL(n,\F_p)$. Note que para la
    fila $k$ de  $A$, hay  $p^n-p^k$ posubles entradas, asi que
    $\ord{GL(n,\F_p)}=\Prod{p^n-p^k}=p^{\frac{n(n-1)}{2}}\Prod{p^j-1}$. Entonces
    cualquier $p$-Sylow de  $GL(n,\F_p)$ tiene orden $p^{\frac{n(n-1)}{2}}$.
\end{example}

\begin{theorem}[El Teorema de Cauchy]\label{8.34}
    S\'i $p$ es un primo y  $p|\ord{G}$, entonces $G$ tiene un elemento de orden
     $p$.
\end{theorem}
\begin{proof}
    Sea $P$ un $p$-Sylow de $G$ y escoja  $g \in P$ tal que  $g \neq e$.
    Entonces $\ord{g}=p^l$ para $l \in \Z^+$. S\'i  $l=1$, terminamos, y s\'i
    $l>1$, note que  $(g^p^{l-1})^p=g^{p^l}=e$.
\end{proof}

\begin{lemma}\label{8.35}
    Sean $H$ y $K$ subgrupos de un grupo $G$. Entonces:
    \begin{equation*}
        \ord{HK}=\frac{\ord{H}\ord{K}}{|H \cap K|}
    \end{equation*}
\end{lemma}
\begin{proof}
    Considere la mapa $f:H \times K \xrightarrow{} HK$ dado por $(h,k)
    \xrightarrow{} hk$. Entonces $f$ es sobre y  $\ord{HK} \leq |H \times K|$.
    Sea entonces $h_1k_1, dots, h_dk_d$ los elementos distintos de $HK$.
    entoncece  $H \times K=\bigcup{\inv{f}(h_ik_i)}$, para todo $1 \leq i \leq
    d$. Ahora,  $\inv{f}(hk)=\{(hk,\inv{g}k) : g \in H \cap K\}$. Entonces
    $|\inv{f}(hk)|=|H \cap K|$. Entonces tenemos que $|H \times
    K|=\ord{H}\ord{K}|H \cap K|=\ord{HK}|H \cap K|$.
\end{proof}

\begin{theorem}[El Segundo Teorema de Sylow]\label{8.36}
    Sea $G$ un grupo finito con orden  $p^rm$ donde $p$ es primo y $p \not| m$.
    Sea  $n_p(G)$ el numero de todos los $p$-subgrupos de Sylow de $G$,
    entonces:
    \begin{equation*}
        n_p(G) \equiv 1 \mod{p}
    \end{equation*}
\end{theorem}
\begin{proof}
    Considere $X=\{P \leq G : P \text{ es } p \text{-Sylow}\}$. Por el primer
    teorema de Sylow, $X \neq \emptyset$. Entonces $|X|=n_p(G)$. Sea que $P \in X$
    actua sobre  $X$ mediante conjugacion. Sea $Q$ ub  $p$-Sylow de  $G$,
    entonces por el teorema \'orbita-estabilizador, tenemos que
    \begin{equation*}
    |\Oc(Q)|=\frac{p^r}{\ord{\stab{Q}}} \in \Z^+
    \end{equation*}
    as\'i que $|\Oc(Q)||p^r$. As'\i que $\Oc(Q)$ tiene largo $1$, o tiene largo
     $p$. Ahora, como
     \begin{equation*}
         |X|=\sum{|\Oc(Q)|}=\sum{|\Oc(Q')|}+\sum{|\Oc(Q'')|}
     \end{equation*}
     donde $Q'$ y  $Q''$ son subgrupos cuyas orbitas tiene $1$ o  $2$ elementos,
     respectivamente, tenemos que  $p|\sum{|\Oc(Q'')|}$, por lo tanto
     \begin{equation*}
         |X| \equiv |\Oc''| \mod{p}
     \end{equation*}
     donde $\Oc''$ es la coleccion de todas las orbitas de largo $1$.

     Ahora, nota que $\Oc(P)-\{P\}$. Suponga entonces que existe un $p$-Sylow
     $Q$ tal que  $g \cdot Q=gQ\inv{g}=Q$ para todo $g \in P$. Entonces,
     $gQ=Qg$, as\'i que  $PQ=QP$ y  $PQ \leq G$. Entonces por el lema de arriba,
     tenemos que
     \begin{equation*}
         \ord{PQ}=\frac{\ord{P}\ord{Q}}{|P \cap Q|}
     \end{equation*}
     Pero $p^r \leq \ord{PQ} \leq p^r$, por lo tanto $Q \subseteq P$. Somo  $P$
     y  $Q$ tienen el mismo orden, tenemos que  $P=Q$, as'\i que  $|\Oc''|=1$
\end{proof}

\begin{theorem}[El Tercer Teorema de Sylow]\label{8.37}
    Sea $G$ un grupo finito con orden  $p^rm$, donde  $p$ es primo y  $p
    \not|m$. Entonces todos los $p$-subgrupos de Sylow son conjugados.
\end{theorem}
\begin{proof}
    Sea $P$ un  $p$-Sylow de  $G$ y  $R$ un  $p$-subgrupo de  $G$. Deje que  $R$
    actua sobre $\faktor{G}{P}$ (no necesariamente el grupo cociente) mediante
    multiplicacion. Por el teorema de Lagrange, tenemos que
    $\ord{\faktor{G}{P}}=[G:P]=\frac{p^rm}{p^r}=m$. Tambien nota que
     $\faktor{G}{P}=\bigcup{\Oc(gP)}$, as\'i que
     \begin{equation*}
         \sum{|\Oc(gP)|}=m
     \end{equation*}
     y existe una orbita cuya longitud no esta dividido por $p$, como  $p \not|
     m$. Por el teorema del \'orbita-establilizador, tenemos que $|\Oc(gP)| |
     \ord{R}=p^l$, para $l \in \Z^+$. As\'i que  $\Oc(gP)$ tiene largo $1$, o
     $p^l$. Ahora, sea  $gP \in \faktor{G}{P}$, un elemento cuya orbita tiene
     largo $1$. Entonces  $g \cdot gP=(hg)P=gP$, para todo $h \in R$, lo que
     dice que  $\inv{g}hg \in P$, por lo tanto $h \in gP\inv{g}$ lo que hace $R
     \subseteq gP\inv{g}$. El resultado entonces se obtiene escogiendo a $R$ un
      $p$-Sylow.
\end{proof}
\begin{corollary}
    Todo $p$-subgrupo de  $G$ est\'a contenido en un  $p$-subgrupo de Sylow.
    Ademas, tenemos que  $n_p(G)|m$
\end{corollary}
