\section*{Lectura 17: Dominios Euclideos}

\begin{definition}
    Sea $R$ un dominio integral. Llamamos a  $R$ un  \textbf{dominio Euclideo}
    s\'i existe una mapa $d:\com{R}{\{0\}} \xrightarrow{} \N$ donde para todo
    $a,b \in R$, existen $q,r \in R$ \'unicos tales que
    \begin{equation*}
        a=qb+r \text{ donde } r=0 \text{ \'o } d(r)<d(b)
    \end{equation*}
\end{definition}

\begin{theorem}\label{17.71}
    S\'i $R$ es un dominio Euclideo, entonces $R$ es un dominio de ideal
    principal.
\end{theorem}
\begin{proof}
    Sea $I$ un ideal de $R$, S\'i $I=(0)$, terminamos; pues, suponga que $I \neq
    (0)$. Considere entonces el conjunto
    \begin{equation*}
        \Dc=\{d(b) : b \in I \text{ y } b \neq 0\}
    \end{equation*}
    Nota que $\Dc \subseteq \N$, as\'i que por el principio de buen orden,
    tenemos que hay un elemento minimo $m \in \Dc$. Sea entonces $b \in I$, con
     $b \neq 0$ tal que  $d(b)=m$. Sea $a \in I$, entonces existen $q,r \in R$
     \'unicas tales que
     \begin{equation*}
         a=qb+r \text{ donde } r=0 \text{ \'o } d(r)<d(b)
     \end{equation*}
     Ahora, note que como $d(b)=m$ es minimo, $d(r) \not{<} d(b)$, mas a\'un,
     tenemos que
     \begin{equation*}
         r=a-qb \in I
     \end{equation*}
     lo que nos dice que $r=0$, y  $a=qb$. Es decir  $I=(a)$.
\end{proof}

\begin{example}\label{}
    \begin{enumerate}
        \item[(1)] Considere $\Z(\sqrt{D})=\{a+b\sqrt{D} : a,b \in \Z\}$, donde
            $D$ no tiene cuadrados. Y sea $d(a+b\sqrt{D})=|a^2-Db^2|$. Note
            que $\Z(\sqrt{D})$ es dominio integral, pues, tome $a,b \neq 0$.
            Considere entonces el cuerpo $\Q(\sqrt{D})$ que contenga a
            $\Z(\sqrt{D})$. Entonces
            \begin{equation*}
                \frac{a}{b}=q' \text{ con } q'=x+y\sqrt{D}, \text{ y } x,y \in \Q
            \end{equation*}
            Sean $x_0, y_0$ tal que $|x-x_0| \leq \frac{1}{2}$ y $|y-y_0| \leq
            \frac{1}{2}$. Tome $q=x_0+y_0\sqrt{D}$ y $r=b((x-x_0)+(y-y_0)\sqrt{D})$.
            Pues, tenemos que
            $q \in \Z(\sqrt{D})$ mas a\'un
            \begin{equation*}
                a=bq+r
            \end{equation*}
             As\'i que $r \in \Z(\sqrt{D})$, y
             \begin{align*}
                 d(r)       &= d(b)d((x-x_0)+(y-y_0)\sqrt{D})  \\
                            &= d(b)|(x-x_0)^2-D(y-y_0)^2|   \\
                            & \leq d(b)|(x-x_0)^2|+|D||(y-y_0)^2|
                            & \leq d(b)(\frac{1}{4}+\frac{|D|}{4})  \\
             \end{align*}
             Pues, s\'i $D=-2,-1,2$ entonces $d(r) \leq d(b)$ y $\Z(\sqrt{D})$ es un
             dominio Euclideo. En el caso de que $D=-1$, poniendo  $i=\sqrt{D}$,
             llamamos a $\Z(i)$ los \textbf{enteros Gaussianos}. Nota que $\Z(i)
             \subseteq \C$.

         \item[(2)] Los enteros $\Z$ son un dominio Euclideano con  $d=|\cdot|$.

         \item[(3)] Para cualquier cuerpo $K$, $K[x]$ es un dominio Euclideano
             con $d(f)=\deg{f}$ para todo $f \in K[x]$.

         \item[(3)] El anillo $\Z[\frac{1+i\sqrt{19}}{2}]$ es un dominio de
             ideal principal, pero no un dominio Euclideano.
    \end{enumerate}
\end{example}
