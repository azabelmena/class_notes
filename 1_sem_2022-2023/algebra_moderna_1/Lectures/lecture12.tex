\section*{Lectura 12: Anillos}

\begin{definition}
    Un \textbf{anillo} $R$ es un grupo abeliano bajo una operacion binaria  $+$
    junto a una operacion binarai $\cdot:(a,b) \xrightarrow{} ab$ tal que
    \begin{enumerate}
        \item[(1)] $\cdot$ es associativa.

        \item[(2)] $a(b+c)=ab+ac$ y $(a+b)c=ac+bc$.
    \end{enumerate}
    S\'i existe un elemento $1 \in R$ tal que  $a_1=1a=a$, entonces llamamos a
    $R$ un anillo con  \textbf{unidad}. Denotamos el elemento de identidad de
    $R$ bajo  $+$ como  $0$. S\'i $ab=ba$ para todo  $a,b \in R$, entonces
    llamamos  $R$  \textbf{commutativa}.
\end{definition}

\begin{definition}
    Sea $R$ un anillo con unidad, y  $a,b \in R$. S\'i $ab=0$ donde  $a\neq 0$,
     y $b \neq 0$, entonces llamamos a $a$ y  $b$  \textbf{divisores de cero}.
     S\' $ab=ba=1$, entonces llamamos a  $a$ y  $b$  \textbf{unidades}.
\end{definition}

\begin{definition}
    Un \textbf{dominio integral} es un anillo commutativa sin divisores de $0$.
    Llamamos la  \textbf{characteristica} de un anillo $R$ de ser el entero mas
    peque\~no $n$ tal que $na=\underbrace{a+\dots+a}_{n-\text{veces}}=0$, para
    todo $a \in R$.
\end{definition}

\begin{definition}
    Sean $R$ y  $S$ anillos. Llamamos a un mapa  $\phi:R \xrightarrow{} S$ un
    \textbf{homomorfismo de anillos} s\'i
    \begin{enumerate}
        \item[(1)] $\phi(a+b)=\phi(a)+\phi(b)$

        \item[(2)] $\phi(ab)=\phi(a)\phi(b)$
    \end{enumerate}
\end{definition}

\begin{example}\label{}
    \begin{enumerate}
        \item[(1)] Sea $\phi:\faktor{\Z}{6\Z} \xrightarrow{} \faktor{\Z}{6\Z}$
            dado por $n \xrightarrow{} 3n$. Entonces $\phi(x+y)=3(x+y)=3x+3y$ y
            $\phi(xy)=3(xy)=3x_3y$, como $3 \cdot 3 \equiv 3 \mod{6}$. As\'i que
            $\phi$ es un homomorfismo de anillos.
    \end{enumerate}
\end{example}

\begin{definition}
    Sea $\phi:R \xrightarrow{} S$ un homomorfismo de anillos. Entonces
    $\ker{\phi}=\{a \in R : \phi(a)=0\}$.
\end{definition}

\begin{lemma}\label{}
    Sea $\phi:R \xrightarrow{} S$ un homomorfismo de anillos, y que los unicos
    idealse de $R$ sean  $(0)$ y $R$. Entonces $\phi$ es 1--1.
\end{lemma}
\begin{proof}
    Nota que $\ker{\phi}$ es ideal, as\'i que $\ker{\phi}=(0)$ o $\ker{\phi}=R$.
    Como $\phi(1)=1$, tenemos que $\ker{\phi} \neq R$.
\end{proof}
