\section*{Lectura 10: El Teorema de Jordan-H\"older}

\begin{definition}
    Sea $G$ un grupo y  $G_0, \dots, G_n$ donde $G_n=\langle e \rangle$ y
    $G_0=G$ tal que $G_{i+1} \unlhd G_i$. Entonces se llama el serie
    \begin{equation*}
        G_n \unlhd \dots \unlhd G_0
    \end{equation*}
    una \textbf{serie subnormal} de $G$.
\end{definition}

\begin{example}\label{}
    \item[(1)] Coje $G_0=D_8$, $D_1=\langle r \rangle$, $G_2=\langle r^2
        \rangle$, $G_3=\langle r^4 \rangle$ y $G_4=\langle e \rangle$. Entonces
        $G_4 \unlhd G_3 \unlhd G_2 \unlhd G_1 \unlhd G_0$.
\end{example}

\begin{definition}
    Sea $G$ un grupo y  $\{G_i\}_{i=0}^n$ una coleccion de subgrupos de $G$
    tales que $G_n=\langle e \rangle$, y $G_{i+1} \unlhd G_i$ son subgrupos
    normales maximales. Entonces la serie subnormal
    \begin{equation*}
        \langle e \rangle=G_n \unlhd \dots \unlhd G_0=G
    \end{equation*}
    se llama una \textbf{serie de composicion} para $G$. Llamamos los factores
    $\faktor{G_i}{G_{i+1}}$ los \textbf{factores} de la serie.
\end{definition}

\begin{lemma}\label{10.43}
    En cualquier serie de composicion, los factores son grupos simples.
\end{lemma}
\begin{proof}
    Esto viene por el teorema de la correspondencia, junto a que los $G_{i+1}
    \unlhd G_i$ son normales maximales.
\end{proof}

\begin{lemma}\label{10.44}
    Sea $G$ un grupo con serie de composicion  $\langle e \langle>=G_n \unlhd
    \dots \unlhd G_0=G$. Para cualquier $K \unlhd G$, removiendo las
    repeticiones de la serie  $\langle e \rangle=K \cap G_n \unlhd \dots \unlhd
    K \cap G_0=K$, obtenemos una serie de composicion para $K$.
\end{lemma}
\begin{proof}
    Sea $x \in K \cap G_i$ y  $g \in K \cap G_{i+1}$. Entonces $xg\inv{x} \in K$
    y $xg\inv{x} \in G_{i+1}$, pues $G_{i+1} \unlhd G_i$. Por lo tanto $K \cap
    G_{i+1} \unlhd K \cap G_i$.

    Ahora miremos a $\faktor{(K \cap G_i)}{(K \cap G_{i+1})}$. Como
    $\faktor{G_i}{G_{i+1}}$ es simple, entonces $G_{i+1}$ es normal maximal en
    $G_i$. Entonces los unicos subgrupos de  $G_i$ que contienen a  $G_{i+1}$ so
    $G_i$ \'o  $G_{i+1}$. Ahora $K \cap G_i \unlhd G_i$, y por lo tanto
    $G_{i+1} \unlhd (K \cap G_i)G_{i+1} \unlhd G_i$. Por lo tanto $G_{i+1}=(K
    \cap G_i)G_{i+1}$, o $G_i=(K \cap G_i)G_{i+1}$. Por el segundo toerema del
    isomorfismo,
    \begin{equation*}
        \faktor{((K \cap G_i)G_{i+1})}{G_{i+1}} \simeq \faktor{(K \cap G_i)}{(K
        \cap G_i \cap G_{i+1})}=\faktor{(K \cap G_i)}{(K \cap G_{i+1})}
    \end{equation*}
    S\'i $G_{i+1}=(K \cap G_i)G_{i+1}$, entonces  $K \cap G_i=K \cap G_{i+1}$ y
    tenemos una repeticion. S\'i $G_i=(K \cap G_i)G_{i+1}$, entonces tenemos que
    $\faktor{G_i}{G_{i+1}} \simeq \faktor{(K \cap G_i)}{(K \cap G_{i+1})}$ y
    terminamos.
\end{proof}

\begin{example}\label{}
    Considere el serie de composicion $\langle 0 \rangle \unlhd \langle 6
    \rangle \unlhd \langle 2 \rangle \unlhd \faktor{\Z}{12\Z}$. Escoja $\langle
    3 \rangle \unlhd \faktor{\Z}{12\Z}$ y obtenemos la serie de composicion para
    $3$ de ser  $\langle 0 \rangle \unlhd \langle 6 \rangle \unlhd \langle 3
    \rangle$.
\end{example}

\begin{theorem}[El Teorema Jordan-H\"older]\label{10.45}
    Sea $G$ un grupo que tiene una serie de composicion. Entonces cualquier dos
    series de composicion para $G$ tiene el mismo largo, mas a\'un s\'i
    \begin{equation*}
        \langle e \rangle=G_n \unlhd \dots \unlhd G_0=G \text{ y }
        \langle e \rangle=H_n \unlhd \dots \unlhd H_0=G
    \end{equation*}
    son series de composiciones para $G$, y  $s \in S_n$ es una permutacion,
    entonces
    \begin{equation*}
        \faktor{G_i}{G_{i+1}} \simeq \faktor{H_{s(i)}}{H_{s(i)+1}}
    \end{equation*}
\end{theorem}

\begin{example}\label{}
    \begin{enumerate}
        \item[(1)] Sea $\langle e \rangle \unlhd \langle r^4 \rangle \unlhd
            \langle r^2 \rangle \unlhd D_8$ Escoja tambien $H=\{e,r^4,t,r^4t\}$
            normal y maximas, entones tenemos que $\langle e \rangle \unlhd
            \langle r^4 \rangle \unlhd H \unlhd D_8$.

        \item[(2)] Sea $\langle 0 \rangle \unlhd \langle 6 \rangle \unlhd
            \langle 2 \rangle \unlhd \faktor{\Z}{12\Z}$ escoja $\langle 0
            \rangle \unlhd \langle 6 \rangle \unlhd \langle 3 \rangle \unlhd
            \faktor{\Z}{12\Z}$ y $\langle 0 \rangle \unlhd \langle 4 \rangle
            \unlhd \langle 2 \rangle \unlhd \faktor{\Z}{12\Z}$.
    \end{enumerate}
\end{example}
