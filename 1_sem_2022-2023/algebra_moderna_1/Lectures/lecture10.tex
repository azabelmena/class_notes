\section*{Lectura 10: El Teorema de Jordan-H\"older}

\begin{definition}
    Sea $G$ un grupo y  $G_0, \dots, G_n$ donde $G_n=\langle e \rangle$ y
    $G_0=G$ tal que $G_{i+1} \unlhd G_i$. Entonces se llama el serie
    \begin{equation*}
        G_n \unlhd \dots \unlhd G_0
    \end{equation*}
    una \textbf{serie subnormal} de $G$.
\end{definition}

\begin{example}\label{}
    \item[(1)] Coje $G_0=D_8$, $D_1=\langle r \rangle$, $G_2=\langle r^2
        \rangle$, $G_3=\langle r^4 \rangle$ y $G_4=\langle e \rangle$. Entonces
        $G_4 \unlhd G_3 \unlhd G_2 \unlhd G_1 \unlhd G_0$.
\end{example}
