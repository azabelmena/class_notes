\section*{Lectura 4: Gurpos Cocientes}

\begin{definition}
    Dado un grupo $G$ y un subgrupo  $H$ de  $G$, definimos el  \textbf{producto
    de clases laterales} de ser el producto $aHbH=abH$.
\end{definition}

\begin{definition}
    Sea $G$ un grupo. Decimos que un subgrupo  $H$ de  $G$ es  \textbf{noraml}
    si para cualquier $a \in G$,  $aH=Ha$. Escribimos $H \unlhd G$.
\end{definition}

\begin{lemma}\label{lemma_4.8}
    Sea $H$ un subgrupo normal de un grupo  $G$. Entonces los siguientes son
    equivalentes para todo $a \in H$:
    \begin{enumerate}
        \item[(1)] $aH\inv{a} \subseteq H$.

        \item[(2)] $aH\inv{a}=H$.

        \item[(3)] Para todo $a \in G$, existe un  $b \in G$ tal que  $aH=Hb$.
    \end{enumerate}
\end{lemma}
\begin{proof}
    S\'i $aH\inv{a}=H$, entonces $aH\inv{a} \subseteq H$. Por el otro lado, si
    $aH\inv{a} \subseteq H$, entonces para $h,h' \in H$,  $ah\inv{a}=h'$, as\'i
    que $h' \in aH\inv{a}$, as\'i que $H \subseteq aH\inv{a}$.

    Ahora, si $aH\inv{a}=H$, entonces tenemos que $aH=Ha$ para todo  $a \in H$,
    por el otro lado, suponga que  $a,b \in H$ tal que  $aH=Hb$. Entonces nota
    que  $a \in Hb$ y  $a \in Ha$, as\'i que  $Ha \cap Hb \neq \emptyset$. Como
     $Ha$ y  $Hb$ son clases de equivalencias, esto forza a  $a=b$.
\end{proof}

\begin{example}\label{}
    $SL(n\R) \unlhd GL(n,\R)$, nota que para cualquier $A \in SL(n,\R)$ y $B \in
    GL(n,\R)$ que $\det{(BA\inv{B})}=(\det{B})(1)(\det{\inv{B}})=1$.
\end{example}

\begin{theorem}\label{thm_4.9}
    S\'i $G$ es un grupo y  $H \unlhd G$ es subgrupo notmal de $G$, entonces las
    clases laterales de $H$ en  $G$ forman un grupo bajo el producto de clases.
\end{theorem}
\begin{proof}
    Define la operaci\'on $(aH,bH) \xrightarrow{} aHbH=\{ahbh' : h,h' \in
    H\}=abH$.  Ya que $aH$ y $bH$ son clases de equivalencia, el producto es
    bien definida.

    Ahora sea  $aH$ y  $bH$, como  $H \unlhd G$, tenemos que  $aHbH=abHH=abH$,
    as\'i que $abH$ es clase lateral de $H$ en  $G$; nota tambien que
    $aH(bHcH)=aH(bcH)=a(bc)H=abcH=(ab)cH=abHcH-(aHbH)cH$, as\'i que el producto
    es associativa.

    Ahora toma la identidad de $H$, $e \in H \unlhd G$ y para cada  $a \in G$,
    toma  $\inv{a}$. Entonces tenemos que $aHeH=aeH=eaH=eHaH=H$ y que  $eH=H$.
    De igual forma  $aH\inv{a}H=a\inv{a}H=\inv{a}aH=\inv{a}HaH=H$. As\'i que $H$
    es la identidad,y  $\inv{a}H$ la inversa de $aH$.
\end{proof}

\begin{definition}
    Sea $G$ un grupo. Denotamos el conjunto de todos clases laterales de un
    subgrupo $H$ en $G$ como $\faktor{G}{H}$. S\'i $H$ es un subgrupo normal,
    entonces  $\faktor{G}{H}$ forma un grupo llamado el \textbf{grupo cociente}
    de $G$ sobre $H$.
\end{definition}

\begin{lemma}\label{lemma_4.10}
    Sea $G$ un grupo. Todo subgrupo de  $G$ es normal s\'i y solo s\'i $H$ es el
    kernel de alg\'un homomorfismo  $\phi$ en  $G$.
\end{lemma}
\begin{proof}
    Sea $H \unlhd G$ Considere la mapa  $\phi:G \xrightarrow{} \faktor{G}{H}$
    tal que $\phi:a \xrightarrow{} aH$. Entonces $\ker{\phi}=\{a \in G :
    aH=h\}$. As\'i que si $a \in \ker{\phi}$, tenemos $aH=H$, que nos dice que
    $a \in H$. Por otro lado,  $a \in H$ implica  $aH=H$, as\'i que  $a \in
    \ker{\phi}$. Es decir $H=\ker{\phi}$.

    Por otro lado considere $\ker{\phi}$ para alg\'un mapa en $G$ Considere
    cualquier $a \in G$ y  $h \in \ker{\phi}$. Entonces
    $\phi(a)\phi(h)\inv{\phi}(a)=\phi(a)e'\inv{\phi}(a)=\phi(a)\inv{\phi}(a)=e'$,
    donde $e'$ es la identidad de $G'$. Entonces como $a$ y $h$ eran
    arbitraros, vemos que $\phi(a)\ker{\phi}\inv{\phi}(a) \subseteq \ker{\phi}$.
    As\'i que $\ker{\phi} \unlhd G$.
\end{proof}

\begin{lemma}\label{lemma_4.11}
    Sea $G$ un grupo y  $\phi:G \xrightarrow{} G'$ un homomorfismo. Entonces
    tenemos que Si $H \unlhd G$ y $\phi$ es sobre, entonces  $\phi(H) \unlhd G'$.
    Mas a\'un s\'i $H' \unlhd G'$, entonces $\inv{\phi}(H') \unlhd G$.
\end{lemma}
\begin{proof}
    Sea $\phi:G \xrightarrow{} G'$ una mapa de $G$ sobre  $G'$. Suponga tambien
    que  $H \unlhd G$. Entonces tome  $y \in G'$. Pues entonces existe un  $x
    \in G$ tal que  $y=\phi(x)$. Tambien existe un $h \in H$ con
    $\alpha=\phi(h)$. Entonces considere
    $y\alpha\inv{y}=\phi(x)\phi(h)\inv{\phi}(y)=\phi(xh\inv{x})=\phi(h')$. Por
    lo tanto $y\alpha\inv{y} \in \phi(H)$ lo que hace $y\phi(H)\inv{y}
    \subseteq \phi(H)$. As\'i que $\phi(H)$ es normal en $G'$.
     Ahora considere $H' \unlhd G'$, entonces para todo  $a' \in G$ y  $h' \in
     H'$,  $a'h'\inv{a'} \in H$. Como $\phi$ es sobre, tenemos que existen $x
     \in G$ y  $h \in H$ con  $x=\phi(a')$ y $h=\phi(h)$, osea $x \in
     \inv{\phi}(G')$ y $h \in \inv{\phi}(H')$. Entonces
     $xh\inv{x}=\phi(a')\phi{h}\inv{\phi}(a')=\phi(a'h\inv{a'}) \in
     \inv{\phi}(H')$. Entonces $x\inv{\phi}(H')\inv{x} \subseteq
     \inv{\phi}(H')$, as\'i que $\inv{\phi}(H')\unlhd G$.
\end{proof}

\begin{theorem}[Teorema del Factor]\label{thm_4.12}
    Suponga que $G$ y $G'$ son grupos y  $H \unlhd H$. Sea $\phi:G
    \xrightarrow{} G'$ y $\pi:G \xrightarrow{} \faktor{G}{H}$ dado por $\pi:a
    \xrightarrow{} aH$. Enotnces existe un u\'unico $\tilde{\phi}:\faktor{G}{H}
    \xrightarrow{} G'$ tal que $\phi=\tilde{\phi} \circ \pi$.
    \[\begin{tikzcd}
	G &&&& {G'} \\
	\\
	\\
	\\
	{\faktor{G}{H}}
	\arrow["\phi", from=1-1, to=1-5]
	\arrow["\pi"', from=1-1, to=5-1]
	\arrow["{\tilde{\phi}}"', dashed, from=5-1, to=1-5]
\end{tikzcd}\]
\end{theorem}
\begin{proof}
    Suponga primero que existe tal $\tilde{\phi}$. Sea $\bar{\phi}:\faktor{G}{H}
    \xrightarrow{} G'$ otro homomorfismo tal que $\phi=\bar{\phi} \circ \pi$.
    Entonces tenemos que $\tilde{\phi} \circ \pi(a)=\bar{\phi} \circ \pi(a)$. Es
    decir que $\tilde{\phi}(aH)=\bar{\phi}(aH)=\phi(a)$. Esto hace que
    $\tilde{\phi}(\faktor{G}{H})=\bar{\phi}(\faktor{G}{H})=\phi(G)$, as\'i que
    tienen el misma imagen y misma relaci\'on. As\'i que
    $\tilde{\phi}=\bar{\phi}$.

    Ahora define la mapa $\tilde{\phi}:\faktor{G}{H} \xrightarrow{} G'$ dado por
    $aH \xrightarrow{} \phi(a)$. Sea entonce_3s $b \in aH$, as\'i que  $aH=bH$,
    entonces tenemos  $\inv{a}b \in H=\ker{\phi}$. Entonces $\phi(\inv{a}b)=e'$,
    la identidad de $G'$, entonces  $\phi(a)=\phi(b)$. Pues $\tilde{\phi}$ esta
    bien definida. Por ultimo, note que
    $\tilde{\phi}(aH)=\tilde{\phi}(\pi(a))=\tilde{\phi} \circ \pi(a)$.
\end{proof}
\begin{corollary}
    $\phi$ es sobre s\'y y solo s\'i $\tilde{\phi}$ es sobre, y $\phi$ es  1--1
    s\'i y solo s\'i $\ker{\phi}=H$.
\end{corollary}
\begin{proof}
    Nota que como $\tilde{\phi}(\faktor{G}{H})=\phi(G)$, entonces s\'i
    $\tilde{\phi}$ es sobre, entonces $\phi$ tiene que ser sobre. Por el otro
    lado, el mismo es cierto.

    Ahora s\'i  $\ker{\phi}=H$, como $H$ es identidad del $\faktor{G}{H}$,
    entonces $\phi$ es 1--1. Por el otro lado, s\'i $\phi$ es 1--1, entonces
    $\ker{\phi}=\vbrack{e_{\faktor{G}{H}}}$, donde $e_{\faktor{G}{H}}$ es la
    identidad de $\faktor{G}{H}$; pero $e_{\faktor{G}{H}}=H$.
\end{proof}
