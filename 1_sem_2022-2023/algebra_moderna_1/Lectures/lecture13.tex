\section*{Lectura 13: Anillos Cocientes}

\begin{definition}
    Sea $R$ un anillo y $I$ un ideal, entonces $\faktor{R}{I}=\{r+I : r \in R\}$
    se llama el \textbf{anillo cociente} de $R$ sobre $I$.
\end{definition}

\begin{lemma}\label{13.56}
    Sea $R$ un anillo, y  $I$ un ideal, entonces el anillo cociente
    $\faktor{R}{I}$ es un anillo bajo la suma de $R$ y la multiplicacion $\cdot$
    dado por $(a+I,b+I) \xrightarrow{} (a+I)(b+I)=ab+I$.
\end{lemma}

\begin{lemma}\label{13.57}
    Todo ideal es el kernel de un homomorfismo de anillos.
\end{lemma}
\begin{proof}
    Escoje $\pi:R \xrightarrow{} \faktor{R}{I}$, entonces $\ker{\pi}=I$.
\end{proof}

\begin{theorem}[El Teorema del Factor]\label{13.58}
    Cualquier homomorfismo de anillis $\phi:R \xrightarrow{} S$ con kernel $K$
    que contiene a un ideal  $I$ se puede factorizarse via  $\faktor{R}{I}$,
    como $\phi=\bar{\phi} \circ \pi$ donde $\pi:R \xrightarrow{} \faktor{R}{I}$
    y $\bar{\phi}:\faktor{R}{I} \xrightarrow{} S$ es el unico homomorfismo con
    $\bar{\phi}$ sobre si y solo si $\phi$ es sobre y  $\bar{\phi}$ 1--1 si y
    solo si $K=I$.

    \[\begin{tikzcd}
	R &&& S \\
	\\
	\\
	{\faktor{R}{I}}
	\arrow["\phi", from=1-1, to=1-4]
	\arrow["{\bar{\phi}}"', from=4-1, to=1-4]
	\arrow["\pi"', from=1-1, to=4-1]
\end{tikzcd}\]

\end{theorem}

\begin{theorem}[Primer Teorema de Isomorfismo]\label{13.59}
    S\'i $\phi:R \xrightarrow{} S$ es un homomorfismo, entonces $\phi(R) \simeq
    \faktor{R}{K}$ donde $K=\ker{\phi}$.
\end{theorem}

\begin{theorem}[Segundo Teorema de Isomorfismo]\label{13.60}
    $\faktor{R+I}{I} \simeq \faktor{R}{R \cap I}$.
\end{theorem}

\begin{example}\label{}
    Considere $\phi:\R[x] \xrightarrow{} \C$ dado por $\phi(p(x))=p(i)$, la
    valuacion de $p$ en $i$. Entonces $\ker{\phi}=\{p(x) : p(i)=0\}=\R[x](x^2+1)$,
    como $i^2+1=0$, tenemos que $\faktor{\R[x]}{(x^2+1)} \simeq \C$.
\end{example}
