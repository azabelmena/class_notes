\section*{Lectura 14: Ideales Maximales y Primos.}

\begin{definition}
    El ideal generado por el conjunto no vacio $X$, de un anillo $R$ escrito
    $I=(X)$, es el ideal mas peque\~no de $R$ tal que  $X \subseteq I$, y se
    llama el ideal  \textbf{generado} por $X$.
\end{definition}

\begin{definition}
    Un ideal \textbf{maximal} en un anillo $R$ es un ideal propio que no esta
    contenido en ningun otro ideal propio. Es decir si $M$ es maximal y  $M
    \subseteq I$, entonces  $M=I$ o  $M=R$.
\end{definition}

\begin{theorem}\label{14.61}
    Sea $M$ un ideal de un anillo commutativo con identidad. Entonces  $M$ es
    maximal si y solo si  $\faktor{R}{M}$ es un cuerpo.
\end{theorem}
\begin{proof}
    Suponga que $M$ es maximal. Entonces como $\faktor{R}{M}$ es un anillo
    commutativo con identidad, sea $a+M \in \faktor{R}{M}$ donde $a+M \neq M$
    note que  $M \subseteq Ra+M$, y como $M$ es maximal, tenemos que $Ra+M=R$.
    Por lo tanto,  $1 \in Ra+M$, y existen  $r, \, \im M$ tales que  $1=ra+m$.
    Note que  $(r+M)(a+M)=ra+M=(1-m)+M=1+M$. Por lo tanto $\inv{(a+M)}=r+M \in
    \faktor{R}{M}$. Esto hace a $\faktor{R}{M}$ un cuerpo.

    Suponga, por otro lado, que $\faktor{R}{M}$ es cuerpo, sea $N$ un ideal tal
    que  $M \subseteq N \subseteq R$, con  $N \neq R$. Considere la mapa  $\pi:R
    \xrightarrow{} \faktor{R}{M}$ dado por $a \xrightarrow{} a+M$. Como $N$ es
    ideal, entonces por el teoream de la correspondencia $\pi(N)$ es un ideal de
    $\faktor{R}{M}$ Por lo tanto $\pi(N)=(0)$, \'o $\pi(N)=\faktor{R}{M}$. Como
    $\pi$ es 1--1, tenemos que  $\pi(N) \neq \faktor{R}{M}$, as\'i que
    $\pi(N)=(0)$. Esto hace $N=M$, $M$ es maximal.
\end{proof}

\begin{definition}
    Un ideal $P$ en un anillo commutativo con identidad es \textbf{primo} s\'i
    para todo $a,b \in R$, s\'i  $ab \in P$ implica que $a \in P$, \'o $b \in P$.
\end{definition}

\begin{theorem}\label{14.62}
    Sea $R$ un anillo commutativo con identidad. Entonce un ideal $P$ de $R$ es
    primo s\'i, y solo s\'i  $\faktor{R}{P}$ es un dominio integral.
\end{theorem}
\begin{proof}
    Suponga que $P$ es primo, y suponga que $(a+P)(b+P)=ab+P=0+P=P$. Entonces
    tenemos que $ab \in P$. Como $P$ es primo, $a \in P$ \'o  $b \in P$, as\'i
    que  $a+P=P$ o  $b+P=P$, es decir, o  $a=0$ \'o  $b=0$.

    Por otro lado, suponga que  $\faktor{R}{P}$ es un dominio integral. Entonces
    $P$ es un ideal propio, es decir,  $P \neq R$, y s\'i  $(a+P)(b+P)=ab+P=P$,
    entonces tenemos que $a+P=P$ \'o  $b+P=P$. Es decir, s\'i  $ab \in P$,
    entonces  $a \in P$ \'o  $b \in P$.
\end{proof}
\begin{corollary}
    Todo ideal maximal es primo.
\end{corollary}
\begin{corollary}
    Sea $\phi:R \xrightarrow{} S$ un homomorfismo de anillos 1--1 con identidad.
    Las siguentes enunciadas son ciertos
    \begin{enumerate}
        \item[(1)] S\'i $S$ es un cuerpo, entonces $\ker{\phi}$ es un ideal
            maximal de $R$.

        \item[(2)] S\'i $S$ es un dominio integral, entonces  $\ker{\phi}$ es un
            ideal primo de $R$.
    \end{enumerate}
\end{corollary}
\begin{proof}
    Por el primer teorema del isomorfismo, nota que $S \simeq
    \faktor{R}{\ker{\phi}}$.
\end{proof}

\begin{example}\label{}
    Sean $\phi:\Z[x] \xrightarrow{} \Z$ dado por $f(x) \xrightarrow{} f(0)$, es
    decir, la mapa valuaci\'on. Entonces tenemos que $\Z \simeq
    \faktor{\Z[x]}{(x)}$.
\end{example}
