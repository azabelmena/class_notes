\section*{Lectura 20: Cuerpos de Descomposici\'on}

\begin{example}\label{}
    \begin{enumerate}
        \item[(1)] Considere $\Q$ junto al polinomio $f(x)=x^2-2$. Note que s\'i
            $f(x)=0$, entonces $x=\sqrt{2}$. Mas a\'un $f$ es irreducible por el
            criterio de Eisentstein para  $p=2$. Por lo tanto
            \begin{equation*}
                \faktor{\Q[x]}{(f)}=\{a+bx : a,b \in \Q \text{ y } x^2=2\} \simeq
                \Q(\sqrt{2})
            \end{equation*}

        \item[(2)] Considere el polinomio $f(x)=x^4-4x^2+1$ sobre $\Q$. S\'i
            $\alpha$ es ra\'iz de $f$, entonces $\alpha^4=4\alpha^2+1$. Mas
            a\'un observe que  $f(x+1)=x^4+4x^3+2x^2-4x-2$, lo cual es
            irreducible, por lo tanto, $f$ es irreducible. Entonces, nota que
            \begin{equation*}
                \faktor{\Q[x]}{(f)} \simeq \Q(\alpha)=
                \{a_0+a_1\alpha+a_2\alpha^2+a_3\alpha^3 : c_i \in \Q \text{ para }
                0 \leq i \leq 3 \text{ y } x^4=4x^2-1\}
            \end{equation*}

        \item[(3)] Sea $f(x)-x^2+2x+1$ sobre el cuerpo $\F_3$. Sea  $\alpha$ una
            ra\'iz de  $f$, entonces  $\alpha^2=\alpha+2$. Observe tambien que
             $f$ no tiene raices en $\F_3$, por lo tanto $f$ es irreducible. Por
              lo tanto, tenemos que
              \begin{equation*}
                  \faktor{\F_3[x]}{(f)} \simeq \F_3(\alpha)=
                  \{a_0+a_1\alpha : a_0,a_1 \in \F_3 \text{ y }
                  \alpha^2=\alpha+2\}
              \end{equation*}
              Observe que s\'i $g(x)=x^2-2$, entonces tenenmos que
              $g(\alpha+1)=(\alpha+1)^2-2=\alpha^2+2\alpha+2=0$, pues
              $\alpha^2+2\alpha+2=f(\alpha)=0$. Por lo tanto, podemos etiquar a
              $\alpha$ como  $\sqrt{2}$ en $\F_3(\alpha)$. Entonces la otra
              ra\'iz de $g$ deber\'ia ser  $-\sqrt{2}=-\alpha-1=2\alpha+2$; y
              $g(2\alpha+2)=0$.
    \end{enumerate}
\end{example}

\begin{definition}
    S\'i $E$ es una extensi\'on de un cuerpo  $F$, y  $f \in F[x]$, entonces
    decimos que $f$ se  \textbf{descompone} sobre $E$ s\'i podemos escribir a
    $f$ de la forma
    \begin{equation*}
        f(x)=c(x-\alpha_1)(x-\alpha_2) \dots (x-\alpha_n)
    \end{equation*}
    donde $c \in F$ y  $\alpha_k \in E$ para todo $1 \leq k \leq n$.
\end{definition}

\begin{definition}
    S\'i $K$ es una extentsi\'on de un cuerpo $F$, y $f \in F[x]$, entonces
    decimos que $K$ es el  \textbf{cuerpo de descomposici\'on} para $f$  s\'i
    $K$ es el cuerpo mas peque\~no por lo cual  $f$ se descompone.
\end{definition}

\begin{example}\label{}
    \begin{enumerate}
        \item[(1)] Considere el polinomio $f(x)=x^2-7$ en $\Q[x]$. Este
            polinomio es irreducible sobre $\Q$. Considere entonces
            $\Q(\sqrt{7})$. Note que $\pm \sqrt{7} \in \Q(\sqrt{7})$. Por lo
            tanto $f$ se descompone sobre $\Q(\sqrt{7})$, de la forma
            \begin{equation*}
                f(x)=(x+\sqrt{7})(x-\sqrt{7})
            \end{equation*}
            Mas a\'un, nota que $[\Q(\sqrt{7}) : \Q]=2$, por lo tanto no existe
            cuerpo entre $\Q(\sqrt{7})$ y $\Q$; lo cual hace a $\Q(\sqrt{7})$ el
            cuerpo de descomosici\'on de $f$.

        \item[(2)]  Considere el polinomio $f(x)=x^3-5$ en $\Q[x]$. Nota que $f$
            es irreducible sobre  $\Q$ y tiene raices  $\sqrt[3]{5}$,
            $\xi\sqrt[3]{5}$, y $\xi^2\sqrt[3]{5}$, donde $\xi^3=1$; es decir,
            $\xi$ es ra\'iz primitiva de $x^3-1$. Note que podemos construir las
            ra\'ices de  $f$ usando  $\xi$ y a  $\sqrt[3]{5}$, por lo tanto, $f$
            se descompone sobre  $\Q(\xi, \sqrt[3]{5})$. Mas a\'un note que
            $\sqrt[3]{5}$ no esta en los cuerpos $\Q(\xi\sqrt[3]{5})$, ni en
            $\Q(\xi^2\sqrt[3]{5})$, as\'i que $f$ no se descompone sobre esos
            cuerpos. Finalemente, not que $\xi$ es ra\'ix de
            $x^3-1=(x-1)(x^2+x+1)$, y como $\xi \neq 1$,  $\xi$ es ra\'iz de
            $x^2+x+1$. Note que  $x^2+x+1$ es irreducible y que  $[\Q(\xi) :
            \Q]=2$, lo que implica que $x^3-5$ no tiene raizes en ese cuerpo, y
             $f$ no se descompone en  $\Q(\xi)$. As\'i que el cuerpo de
             descompoaici\'on de $f$ es  $\Q(\xi,\sqrt[3]{5})$.
             \[\begin{tikzcd}
                & {\Q(\xi,\sqrt[3]{5})} \\
                & {\Q(\sqrt[3]{5})} & {\Q(\xi\sqrt[3]{5})} & {\Q(\xi^2\sqrt[3]{5})} \\
                {\Q(\xi)} \\
                & {\Q}
                \arrow[no head, from=1-2, to=2-2]
                \arrow[no head, from=1-2, to=2-3]
                \arrow[no head, from=1-2, to=2-4]
                \arrow[no head, from=1-2, to=3-1]
                \arrow[no head, from=3-1, to=4-2]
                \arrow[no head, from=2-2, to=4-2]
                \arrow[no head, from=2-3, to=4-2]
                \arrow[no head, from=2-4, to=4-2]
             \end{tikzcd}\]
             Note que este reticulo de cuerpos es isomorfo al reticulo de los
             grupos $S_3$ y $D_3$.

         \item[(2)] Considere $f(x)=x^2+2$ sobre $\Q$. Tenemos las ra\'ices de
             $f$ dado por  $\pm i\sqrt{2}$, donde $i^2=-1$. Tenemos que  $f$ se
             descompone sobre $\Q(i,\sqrt{2})$. Mas a\'un, los cuerpos entre
             $\Q(i,\sqrt{2})$ est\'an dados por el reticulo
             \[\begin{tikzcd}
                & {\Q(i,\sqrt{2})} \\
                \Q(i) & \Q(i\sqrt{2}) & \Q(\sqrt{2}) \\
                & {\mathbb{Q}}
                \arrow[no head, from=1-2, to=2-2]
                \arrow[no head, from=1-2, to=2-1]
                \arrow[no head, from=2-1, to=3-2]
                \arrow[no head, from=3-2, to=2-3]
                \arrow[no head, from=2-3, to=1-2]
                \arrow[no head, from=2-2, to=3-2]
              \end{tikzcd}\]
              Pero $f$ se descompone sobre  $\Q(i\sqrt{2})$, por lo tanto
              $\Q(i,\sqrt{2})$ no es el cuerpo de descomposici\'on para $f$;
              sino es  $\Q(i\sqrt{2})$. Note que el retriculo de estos cuerpos
              es isomorfo al reticulo de $\faktor{\Z}{2\Z} \times
              \faktor{\Z}{2\Z}$, a $\Hb$ los cuaterniones, y a  $V_4$, el grupo
              $4$ de Klein.

          \item[(4)] Considere el polinomio $f(x)=x^4+x^2+1$ sobre $\F_5$.
              Observer que
              \begin{equation*}
                  f(x)=(x^2+x+1)(x^2+4x+1)
              \end{equation*}
              Sea $\alpha$ una ra\'iz de $x^2+x+1$ y considere $\F_5(\alpha)$.
              Sean $f_1(x)=x^2+x+1$ y $f_2(x)=x^2+4x+1$. Observer que
              $f_1(4\alpha+4)=\alpha^2+\alpha+1=0$, as\'i que $f_1$ se
              descompone sobre $\F_5(\alpha)$. Similarmente $f_2(\alpha)=0$, y
              $f_2$ se descompone sobre $\F_5(\alpha)$. Entonces $f$ se
              descompone sobre  $\F_5(\alpha)$; mas a\'un, $[\F_5(\alpha) :
              \F_5]=2$ por lo tanto no existe cuerpo entre $\F_5(\alpha)$ y
              $\F_5$, por lo tanto $\F_5(\alpha)$ es el cuerpo de
              descomposic\'on de $f$.
    \end{enumerate}
\end{example}

\begin{lemma}\label{lemma_84}
    S\'i $f$ es un polinomio con grado  $\deg{f}=n$, entonces $f$ tiene un
    cuerpo de descomposici\'on  $K$ sobre  $F$ tal que  $[K:F] \leq n!$.
\end{lemma}
\begin{proof}
    Asuem que $n \geq 1$. Entoces  $f$ tiene al menos una raiz  $\alpha_1$. Como
    $f(\alpha_1)=0$, entonces el polinomio minimo de $\alpha$,  $m_1$ divide a
    $f$; es decir, que $f(x)=m_1(x)f_1(x)$ para alg\'un $f_1 \in F[x]$. Como
    $\deg{m_1} \leq n$, tenemos que $[F(\alpha):F] \leq n$.

    Ahora, escribe $f(x)=(x-\alpha_1)^{r_1}g(x)$, donde $g \neq 0$ y  $\deg{g}
    \leq n-1$. S\'i $g$ es constante, terminamos. Al contrario, sea  $\alpha_2$
    ra\'iz de $g$, entonces es ra\'iz de  $f$. Sea  $m_2$ el polinomio minimo de
    $\alpha_2$, entonces $m_2|g$, es decir, $g(x)=m(x)g_2(x)$, y como $\deg{m_2}
    \leq n-1$, tenemos que $[F(\alpha_1,\alpha_2):F] \leq n(n-1)$.

    Siguiendo este proceso, sea $\alpha_1, \dots, \alpha_m$ raizes de $f$,
    entonces $f(x)=(x-\alpha_1)^{r_1} \dots (x-\alpha_m)^{r_m}$ y tiene cuerpo
    de descomopsici\'on $F(\alpha_1, \dots, \alpha_m)$ con $[F(\alpha_1, \dots,
    \alpha_m) : F] \leq n!$.
\end{proof}

\begin{theorem}\label{theorem_85}
    S\'i $\alpha$ y  $\beta$ son raices de un polinomio irreducible $f \in
    F[x]$ en una extensi\'on $E$, entonces  $F(\alpha) \simeq F(\beta)$. Es
    decir los cuerpos de descomposici\'on son \'unicos hasta el isomorfismo.
\end{theorem}
\begin{proof}
    Suponga que $f$ es m\'onico. Como  $f(\alpha)=f(\beta)=0$, tenemos que $f$
    es polinomio minimo de  $\alpha$ y de  $\beta$. Digamos que  $\deg{f}=n$,
    s\'i $a \i F(\alpha)$, entonces tenemos
    \begin{equation*}
        a=a_0+a_1\alpha+\dots+a_{n-1}\alpha^{n-1}
    \end{equation*}
    Pues, considere el mapa $a \xrightarrow{} b$, donde
    \begin{equation*}
        b=b_0+b_1\beta+\dots+b_{n-1}\beta^{n-1}
    \end{equation*}
    que es un isomorfismo entre $F(\alpha)$ y $F(\beta)$.
\end{proof}

\begin{example}\label{}
    \begin{enumerate}
        \item[(1)] Sea $x^3-2 \in \Q[x]$ con raiz $\sqrt[3]{2},
            \xi\sqrt[3]{2}$, y $\xi^2\sqrt[3]{2}$, donde  $\xi^3=1$. Entonces el
            cuerpo de descomposici\'on de $x^3-2$ sobre $\Q$ es
            $\Q(\sqrt[3]{2}, i\sqrt{3})$ con $[\Q(\sqrt[3]{2}, i\sqrt{3})]=6=3!$.
            Aqui $\xi$ es la tercera raiz unitario primitiva.
    \end{enumerate}
\end{example}
