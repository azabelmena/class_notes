\section*{Lectura 18: Polinomios Irreducibles}

En el caso de un cuerpo $F$, las unidades de  $F[x]$ son las unidades de $F$ no
cero y los polinomios irreducibles son aquellos de grado $\deg=1$ o grado
$\deg>1$ que no se puede factorizar en polinomios de grado menor.

\begin{example}\label{}
    $4x+2$ es irreducible en  $\Q$, pero $4x+2=2(2x+1)$ en $\Z$.
\end{example}

\begin{theorem}\label{18.72}
    Sea $R$ un dominio integral, y defina la relaci\'on de equivalencia $\sim$
    sobre $R \times \com{R}{\{0\}}$ dado por
    \begin{equation*}
        (a,b) \sim (c,d) \text{ s\'i y solo s\'i } ad-bc=0
    \end{equation*}
    Sea $Q=\faktor{R \times \com{R}{\{0\}}}{\sim}$ el conjunto factor y defina
    las operaciones $+$ y  $\cdot$ dados por
    \begin{equation*}
        (a,b)+(c,d) &=  (ad+bc,bd)  \\
        (a,b)(c,d)  &= (ac,bd)  \\
    \end{equation*}
    Entonces $Q$ forma un cuerpo bajo estos operaciones.
\end{theorem}
\begin{proof}
    Nota que $(Q,+)$ forma un grupo abeliano con identidad $(0,1)$ y inversos
    $(-a,b)$. De igaul forma, $(Q,\cdot)$ forma un grupo abeliano con la
    identidad $(1,1)$ y inversos $(b,a)$. Por ultimo, note que
    \begin{equation*}
        (a,b)((c,d)+(e,f))=(a,b)(cf+de,df)=(acf+ade,bdf)=(a,b)(c,d)+(a,b)(e,f)
    \end{equation*}
\end{proof}

\begin{definition}
    Sea $R$ un dominio integral y considera la relaci\'on de equivalencia
    $\sim$ dado sobre $R \times \com{R}{\{0\}}$ por
    \begin{equation*}
        (a,b) \sim (c,d) \text{ s\'i y solo s\'i } ad-bc=0
    \end{equation*}
    Entonces llamamos al cuerpo $Q=\faktor{R \times \com{R}{\{0\}}}{\sim}$ el
    \textbf{cuerpo de fracciones} sobre $R$.
\end{definition}

\begin{example}\label{}
    \begin{enumerate}
        \item[(1)] El cuerpo de fracciones de $\Z(\sqrt{D})$ es precisamente el
            cuerpo $\Q(\sqrt{D})$.

        \item[(2)] El cuerpo de fracciones de $\Z$ es $\Q$.
    \end{enumerate}
\end{example}
