\section*{Lectura 7: Acciones de Grupos.}

\begin{theorem}[EL Teorema de Cayley]\label{thm_7.26}
    Todo grupo es isomorfo a un subgrupo del grupo de simetrico.
\end{theorem}
\begin{proof}
    Sea $G$ un grupo y  $A(G)$ el grupo simetrico de $G$. Defnia $\lambda:G
    \xrightarrow{} A(G)$ dado por $g \xrightarrow{} \lambda_g$, donde
    $\lambda_g:G \xrightarrow{} G$ esta dado por $x \xrightarrow{} gx$. Note que
    $\lambda_g$ es un permutacion de los elementos de  $G$, es sobre, y es 1--1
    por cancelacion, as\'i que  $\lambda_g \in A(G)$. As\'i que $\lambda$ es
    bien definido.

    Ahora suponga que que  $\lambda(g)=\lambda(h)$, entonces para alg\'un $x \in
    G$,  $\lambda_g(x)=\lambda_g(h)$, pues $gx-gh$. Por cancelaci\'on, tenemos
    que  $g=h$. s\'i que  $\lambda$ es 1--1. Ahora dado  $x \in G$, que
    $\lamda(gh)(x)=\lambda_{gh}(x)=(gh)x=g(hx)=g(\lamda_h(x))=\lambda_g(\lambda_h(x))=
    \lambda_G\lambda_h(x)$. As\'i que $\lambda$ definia  una isomorfismo de $G$
    hac\'ia  $\lambda(G)$ lo cual es subgrupo de $A(G)$.
\end{proof}

\begin{example}\label{}
    Por la teorema de Cayley, tenemos que $D_3 \simeq S_6$.
\end{example}

\begin{definition}
    Un grupo $G$  \textbf{actua} sobre un conjunto $X$ s\'i para todo  $g \in
    G$, existe una mapa  $G \times X \xrightarrow{} X$ dado por $(g,x)
    \xrightarrow{} g \cdot x$ tal que:
    \begin{enumerate}
        \item[(1)] $h \cdot (g \cdot x)=(hg) \cdot x$.

        \item[(2)] $e \cdot x=x$ para todo  $x \in X$.
    \end{enumerate}
\end{definition}

\begin{example}\label{}
    \begin{enumerate}
        \item[(1)] Todo grupo actua sobre si mismo bajo multiplicacion pr la
            izquierda. Llamamos esto el \textbf{accion regular}.

        \item[(2)] Todo grupo actua sobre si mismo via la accion de
            \textbf{conjugacion} definido pro $(g,x) \xrightarrow{} gx\inv{g}$.
            Nota que $h \cdot (g \cdot x)=h \cdot
            (gx\inv{g})=hgx\inv{g}\inv{h}=(hg)x\inv{(hg)}=(hg) \cdot x$. Tambein
            $e \cdot x=ex\inv{e}=x$.
    \end{enumerate}
\end{example}

\begin{definition}
    Definimos el \textbf{kernel} de una accion $G \times X \xrightarrow{} X$ de
    ser el conjunto $\ke=\{g \in G: g \cdot x=x\}$.
\end{definition}

\begin{example}\label{}
    \begin{enumerate}
        \item[(1)] S\'i $G$ actua sobre si mismo via conjugacion, entonces si
            $gx\inv{g}=x$,
            tenemos que $gx=xg$ para todo  $x \in G$. Por lo tanto  $\ker=\{g \in
                G : gx=xg \text{ para todo } x \in G\}$. Llamamos este kernel el
                \textbf{centro} de $G$, y lo denotamos como  $Z(G)$.

            \item[(2)] Conisdere $\Bc_n$ el conjunto de todos funciones
                booleanas  $f:\F_2^n \xrightarrow{} \F_2^n$ en $n$ variables.
                Defina una operacion de $S_n$ sobre  $\Bc_n$ definida por  $s
                \cdot f(x_1, \dots, x_n)=f(x_{s(1)}, \dots, x_{s(n)})$. Este
                operaci\'on defina una acci\'on de grupos de $S_n$ sobre
                $\Bc_n$. Nota que el kernel de este accion es trivial.
    \end{enumerate}
\end{example}

\begin{definition}
    Sea $G$ un grupo que actua sobre un conjunto $X$. La \textbf{\'orbita} de un
    $x \in X$ es el conjunto  $\Oc(x)=\{g \cdot x : g \in G\}$.
\end{definition}

\begin{example}\label{}
    \begin{enumerate}
        \item[(1)] Sea $G$ un grupo actuando sobre si mismo por su
            multiplicacion  (por izquierda). Suponga que $x \in G$ y sea  $g \in
            G$ un elemento cualquiera. Entonces existe un  $g_0 \in G$ tal que
            $g=g_0x$. Esto hace $G \subseteq \Oc(x)$. Por lo tanto $\Oc(x)=G$.

        \item[(2)] Considere un grupo $G$ actuando sobre si mismo mediante
            conjugacion. Sea $x \in G$. Entonces $\Oc(x)=\{gx\inv{g} : g \in
            G\}=\cl{x}$. Llamamos a $\cl{x}$ la \textbf{clase de conjugacion} de
            $x$.

        \item[(3)] Considere $\Bc_3$ y defina
            $f(x_1,x_2,x_3)=x_1+x_2x_3+x_1x_2x_3$. Sea $S_3=\{(1), (2 \ 3), (1 \
            2), (1 \ 2 \ 3), (1 \ 3 \ 2), (1 \ 3)\}$. Entonces:
            \begin{align*}
                (1) \cdot f     &=  x_1+x_2x_3+x_1x_2x_3=f  \\
                (2 \ 3) \cdot f         &= x_1+x_3x_2+x_1x_3x_2=f    \\
                (1 \ 2) \cdot f     &= x_2+x_1x_3+x_2x_1x_3=f_1     \\
                (1 \ 2 \ 3) \cdot f     &=  x_2+x_3x_1+x_3x_2x_1=f_1    \\
                (1 \ 3 \ 2) \cdot f     &=  x_3+x_1x_2+x_3x_1x_2=f_2    \\
                (1 \ 3) \cdot f     &=  x_3+x_2x_1+x_3x_2x_1=f_2    \\
            \end{align*}
            As\'i que $\Oc(f)=\{f,f_1,f_2\}$. Nota que $|\Oc(x)|$ divide a
            $\ord{S_3}$.
    \end{enumerate}
\end{example}

\begin{lemma}\label{}
    Sea $G$ un grupo que actua sobre un conjunto  $X$. Entonces las \'orbitas de
     $X$ particionan a  $X$.
\end{lemma}
\begin{proof}
    Sea $x \in \Oc(y)$ y $x \in \Oc(z)$ para $x,y,z \in X$. Entonces vemos que
    $x=gy$ y  $x=hz$, por lo tanto  $gy=hz$. Es decir  $y=(\inv{g}h)z$, por lo
    tanto $y \in \Oc(z)$. De igual forma, $z \in \Oc(y)$. Esto hace que
    $\Oc(y)=\Oc(y)$.
\end{proof}

\begin{definition}
    Sea $G$ un grupo actuando sobre un conjunto  $X$. El  \textbf{estabilizador}
    de $x \in X$ es el conjunto  $\stab{x}=\{g \in G : g \cdot x=x\}$.
\end{definition}

\begin{lemma}\label{7.28}
    Sea $G$ un grupo que actua sobre un conjunto $X$. Entonces el estabilizador
    de todo $x \i X$ es subgrupo de  $G$.
\end{lemma}
\begin{proof}
    Sea $x \in X$ y sea $g,h \in \stab{x}$. Entonces $x=gx$ y  $x=\inv{h}x$. Por
    lo tanto $(g\inv{h}) \cdot x=x$.
\end{proof}

\begin{example}\label{}
    Para cualquier grupo actuando sobre si mismo bajo conjugacion,
    $\stab{x}=\{g : gx=xg\}=C(x)$ que se llama el \textbf{centralizador} de $x$.
\end{example}

\begin{theorem}[Teorema del \'Orbita-Estabilizador.]\label{7.29}
    Suponga que $G$ es un grupo que actua sobre un conjunto  $X$. Sean  $
    \Oc(x)$ y $\stab{x}$ la \'orbita y estabilizador de un $x \in X$. Entonces:
    \begin{equation*}
        |\Oc(x)|=[G:\stab{x}]
    \end{equation*}
\end{theorem}
\begin{proof}
    Suponga que $y \in \Oc(x)$. Entonces $y=g \cdot x$ para alg\'un  $g \in G$.
    Defina ahora la mapa  $f:\Oc(x) \xrightarrow{} \faktor{G}{\stab{G}}$ dado
    por $y=g \cdot x \xrightarrow{} g\stab{x}$. Sea ahora $y=g \cdot x=h \cdot
    x$. Entonces vemos que  $x=(\inv{g}h) \cdot x$, as\'i que $\inv{g}h \in
    \stab{x}$. Esto hace que $g\stab{x}=h\stab{x}$. Por lo tanto $f$ es bien
    definida.

    Ahora, vemos que $f$ es sobre; s\'i $y \in \Oc(x)$, entonces $y=g \cdot x$
    para alg\'un $g \in G$, as\'i que a cada  $y \in \Oc(x)$ est\'a asignada a
    un $g\stab{x}$. M\'as aun, $f$ es 1--1. Sean $y=gx$ y  $y'=hx$. S\'i
    $g\stab{x}=h\stab{x}$, entonces $\inv{g}h \in \stab{x}$, as\'i que $gx=hx$,
    es decir  $y=y'$. Por lo tanto, tenemos una mapa 1--1 de  $\Oc(x)$ sobre el
    conjunto $\faktor{G}{\stab{x}}$, que tiene cardinalidad $[G:\stab{x}]$.
\end{proof}
\begin{corollary}
    S\'i $G$ es un grupo finito, entonces  $|\Oc(x)|$ divida a $\ord{G}$. En
    particular
    \begin{equation*}
        |\Oc(x)|=\frac{\ord{G}}{\ord{(\stab{x})}}
    \end{equation*}
\end{corollary}
