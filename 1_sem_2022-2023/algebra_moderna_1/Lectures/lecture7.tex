\section*{Lectura 7: Acciones de Grupos.}

\begin{theorem}[EL Teorema de Cayley]\label{thm_7.26}
    Todo grupo es isomorfo a un subgrupo del grupo de simetrico.
\end{theorem}
\begin{proof}
    Sea $G$ un grupo y  $A(G)$ el grupo simetrico de $G$. Defnia $\lambda:G
    \xrightarrow{} A(G)$ dado por $g \xrightarrow{} \lambda_g$, donde
    $\lambda_g:G \xrightarrow{} G$ esta dado por $x \xrightarrow{} gx$. Note que
    $\lambda_g$ es un permutacion de los elementos de  $G$, es sobre, y es 1--1
    por cancelacion, as\'i que  $\lambda_g \in A(G)$. As\'i que $\lambda$ es
    bien definido.

    Ahora suponga que que  $\lambda(g)=\lambda(h)$, entonces para alg\'un $x \in
    G$,  $\lambda_g(x)=\lambda_g(h)$, pues $gx-gh$. Por cancelaci\'on, tenemos
    que  $g=h$. s\'i que  $\lambda$ es 1--1. Ahora dado  $x \in G$, que
    $\lamda(gh)(x)=\lambda_{gh}(x)=(gh)x=g(hx)=g(\lamda_h(x))=\lambda_g(\lambda_h(x))=
    \lambda_G\lambda_h(x)$. As\'i que $\lambda$ definia  una isomorfismo de $G$
    hac\'ia  $\lambda(G)$ lo cual es subgrupo de $A(G)$.
\end{proof}

\begin{example}\label{}
    Por la teorema de Cayley, tenemos que $D_3 \simeq S_6$.
\end{example}

\begin{definition}
    Un grupo $G$  \textbf{actua} sobre un conjunto $X$ s\'i para todo  $g \in
    G$, existe una mapa  $G \times X \xrightarrow{} X$ dado por $(g,x)
    \xrightarrow{} g \cdot x$ tal que:
    \begin{enumerate}
        \item[(1)] $h \cdot (g \cdot x)=(hg) \cdot x$.

        \item[(2)] $e \cdot x=x$ para todo  $x \in X$.
    \end{enumerate}
\end{definition}

\begin{example}\label{}
    \begin{enumerate}
        \item[(1)] Todo grupo actua sobre si mismo bajo multiplicacion pr la
            izquierda. Llamamos esto el \textbf{accion regular}.

        \item[(2)] Todo grupo actua sobre si mismo via la accion de
            \textbf{conjugacion} definido pro $(g,x) \xrightarrow{} gx\inv{g}$.
            Nota que $h \cdot (g \cdot x)=h \cdot
            (gx\inv{g})=hgx\inv{g}\inv{h}=(hg)x\inv{(hg)}=(hg) \cdot x$. Tambein
            $e \cdot x=ex\inv{e}=x$.
    \end{enumerate}
\end{example}

\begin{definition}
    Definimos el \textbf{kernel} de una accion $G \times X \xrightarrow{} X$ de
    ser el conjunto $\ke=\{g \in G: g \cdot x=x\}$.
\end{definition}

\begin{example}\label{}
    S\'i $G$ actua sobre si mismo via conjugacion, entonces si  $gx\inv{g}=x$,
    tenemos que $gx=xg$ para todo  $x \in G$. Por lo tanto  $\ker=\{g \in G :
    gx=xg \text{ para todo } x \in G\}$. Llamamos este kernel el \textbf{centro}
    de $G$, y lo denotamos como  $Z(G)$.
\end{example}
