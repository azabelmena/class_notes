\section*{Lectura 11: Grupos Resolubles y Nilpotentes}

\begin{definition}
    Sea $G$ un grupo. Un subgrupo $H$ de $G$ se llama \textbf{characteristica}
    s\'i para cada automorfismo $\phi$ de $G$, $\phi(H)=H$; es decir que $\phi$
    restringido a $H$ es sobre. Escribimos $H \Char{G}$
\end{definition}

\begin{lemma}\label{11.46}
    Sea $G$ un grupo y  $H \leq K \leq G$ subgrupos. Entonces
    \begin{enumerate}
        \item[(1)] S\'i $H \Char{K}$ y $K \Char{G}$, entonces $H \Char{G}$.

        \item[(2)] S\'i $H \Char K$ y  $K \unlhd G$, entonces  $H \unlhd G$.
    \end{enumerate}
\end{lemma}
\begin{proof}
    Suponga que $H \leq K \leq G$. Sea  $\phi$ un automorfismo de $G$, entonces
    $\phi(K)=K$. Es decir que $\phi'=\phi|_K$ es sobre. Entonces vemos tambien
    que $\phi'(H)=H$, pero $\phi'(H)=\phi(H)$, as\'i que $H \Char{G}$.

    Ahora considere el automorfismo de $K$ dado por  $k \xrightarrow{}
    gk\inv{g}$ para $g \in G$. Para cualquier $g$ tenemos un automorfismo bien
    definido de $K$. Por lo tanto esta preserva a  $H$, como  $H \Char{K}$, es
    decir que $gH\inv{g}=H$.
\end{proof}

\begin{definition}
    El \textbf{subgrupo commutador} $G'$ de un grupo $G$ es el subgrupo de $G$
    generado por todos los elements commutadores de $G$,
    $[x,y]=xy\inv{x}\inv{y}$. Tambien llamamos a $G'$ la  \textbf{derivada} de
    $G$.
\end{definition}

\begin{lemma}\label{11.47}
    El subgrupo commutador de un grupo verdaderamente es un subgrupo.
\end{lemma}

\begin{lemma}\label{11.48}
    Sea $G'$ el commutador de $G$. Entonces los sigueintes enunciados son
    ciertos.
    \begin{enumerate}
        \item[(1)] $G' \Char G$.

        \item[(2)] S\'i $G$ es abeliano, entonces  $G'=\langle e \rangle$.

        \item[(3)] $\faktor{G}{G'}$ es abeliano.

        \item[(4)] S\'i $N \unlhd G$, entonces  $\faktor{G}{N}$ es abeliano s\'i
            y solo s\'i $G' \leq N$.
    \end{enumerate}
\end{lemma}
\begin{proof}
    \begin{enumerate}
        \item[(1)] Sea $\phi \in \Aut{G}$, entonces $\phi([x,y])
            =\phi(xy\inv{x}\inv{y})=\phi(x)\phi(y)\inv{\phi}(x)\inv{\phi}(y)
            =[\phi(x),\phi(y)]$. As\'i que $\phi(G')=G$.

        \item[(2)] Suponga que $G$ es abeliano, entonces para todo  $[x,y] \in
            G'$, $xy\inv{x}\inv{y}=x\inv{x}y\inv{y}=e$.

        \item[(3)] Como $G' \unlhd G$,  $\faktor{G}{G'}$ es un grupo. Ahora,
            sean $xG',yG' \in \faktor{G}{G'}$, entonces $xG'yG'=xyG'$ lo que
            dice que  $xy\inv{x}\inv{y}=(xy)\inv{(yx)} \in G'$, entonces
            $(xy)G'=(yx)G'$.

        \item[(4)] Por ultimo, s\'i $N \unlhd G$ y  $\faktor{G}{N}$ es abeliano,
            entonces $xNyN=xyN=yxN=yNxN$, lo que dice que  $(xy)\inv{(yx)} \in
            N$, lo que dice $[x,y] \in N$; as\'i que $G' \leq N$. Por otro lado,
            s\'i  $G' \leq N$, entonces  $[x,y]=(xy)\inv{(yx)} \in N$ lo que
            dice que $xyN=yxN$.
    \end{enumerate}
\end{proof}
\begin{corollary}
    $\faktor{G}{G'}$ es el grupo abeliano mas grande que se puede formar por
    factores.
\end{corollary}

\begin{lemma}\label{11.49}
    S\'i $G$ es un grupo, y  $H \leq G$ un subgrupo de  $G$ entonces  $H' \leq
    G'$.
\end{lemma}
\begin{proof}
    Como $H \leq G$,  $x,y,g,h \in H$ implica $(xg)(yh)\inv{(xg)}\inv{(yh)} \in
    H$, as\'i que $[xg,yh] \in H'$ cuando $[x,y],[g,h] \in H'$. Mas a\'un s\'i
    $[x,y] \in H'$, entonces $xy\inv{x}\inv{y} \in H$, as\'i que
    $\inv{y}\inv{x}yx \in H$ entonces $[\inv{y},\inv{x}] \in H'$.
\end{proof}

\begin{definition}
    Sea $G$ un grupo. Para cualquier $n \in \N$, definimos recursivamente el
    \textbf{$n$-esima derivada} de $G$ como:
    \begin{enumerate}
        \item[(1)] $G^{(0)}=G$ y $G^{(1)}=G'$.

        \item[(2)] $G^{(n+1)}=(G^{(n)})'$ para todo $n \geq 0$.
    \end{enumerate}
\end{definition}

\begin{definition}
    Llamamos una serie subnormal $\langle e \rangle=G_n \unlhd \dots \unlhd
    G_0=G$ una \textbf{serie normal} s\'i para todo $0 \leq i \leq j \leq n$,
    tenemos $G_j \unnlhd G_i$.
\end{definition}

\begin{definition}
    Un grupo $G$ se llama \textbf{resoluble} s\'i en algun momento la $n$-esima
    derivada de $G$ es trivial para alg\'un $n \geq 0$. Mas precisamente,
    existe una serie normal
    \begin{equation*}
        \langle e \rangle=G^{(n)} \unlhd G^ {(n-1)} \unlhd \dots \unlhd G^{(0)}=G
    \end{equation*}
\end{definition}

\begin{lemma}\label{11.50}
    Todo grupo abeliano es resoluble.
\end{lemma}
\begin{proof}
    Por supuesto, s\'i $G$ es un grupo abeliano, entonces $G'=\langle e \rangle$
    lo cuale es la $1$-esmia derivada. Pues  $G$ tiene el serie normal  $\langle
    e \rangle=G^{(1)}=G' \unlhd G^{(0)}=G$.
\end{proof}
\begin{corollary}
    $G$ es un grupo simple y resoluble s\'i  $G$ es ciclico de orden  $p$,  $p$
    un primo.
\end{corollary}
\begin{proof}
    Con $G$ simple y resoluble. Entonces los unicos subgrupos normales de  $G$
    so  $\langle e \rangle$ y si mismo, as\'i que $G'=G$ o  $G'=\langle e
    \rangle$. Pero como $G$ es resoluble,  $G' \neq G$, al contrario $G^{(n)}=G$
    para todo $n \geq 0$ seria cierto. Por lo tanto $G$ es abeliano, lo que dice
    que  $G \simeq \faktor{\Z}{p\Z}$ para $p$ primo.
\end{proof}
\begin{corollary}
    Un grupo noabeliano y simple no puede ser resoluble.
\end{corollary}
\begin{proof}
    Al no ser abeliano, tenemos $G' \neq \langle e \rangle$, as\'i que $G'=G$.
\end{proof}

\begin{theorem}\label{11.51}
    Las siguientes enunciados son equivalentes.
    \begin{enumerate}
        \item[(1)] $G$ es un grupo resoluble.

        \item[(2)] $G$ tiene una serie normal
            \begin{equation*}
                \langle e \rangle=G_n \unlhd \dots \unlhd G_0=G
            \end{equation*}
            con todos los factores abelianas.

        \item[(3)] $G$ tiene una serie subnormal
            \begin{equation*}
                \langle e \rangle=G_n \unlhd \dots \unlhd G_0=G
            \end{equation*}
            con todos los factores abelianas.
    \end{enumerate}
\end{theorem}
\begin{proof}
    Ciertamente, s\'i $G$ es resoluble, entonces la serie $\langle e
    \rangle=G^{(n)} \unlhd \dots \unlhd G^{(0)}=G$ es una serie normal cuyas
    factores son abelianas. Ademas, de esto ser cierto, tenemos que todo serie
    normal es subnormal; as\'i que $\langle e \rangle=G_n \unlhd \dots \unlhd
    G_0=G$ es una serie subnormal con los factores abelianas.

    Ahora, suponga que $\langle e \rangle=G_n \unlhd \dots \unlhd G_0=G$ es una
    serie subnormal donde $\faktor{G_i}{G_{i+1}}$ es abeliana para todo $0 \leq
    i \leq n-1$. Para  $i=0$, tenemos que  $G_1 \unlhd G$ y  $\faktor{G}{G_1}$
    es abeliano, por lo tanto $G'=G^{(1)} \leq G_1$. Por inducci\'on, suponga
    que para todo $i \geq 0$ que  $G^{(i)} \leq G_i$. Como
    $G^{(i+1)}=(G^{(i)})'$, por hipotesis tenemos que $G^{(i+1)} \unlhd G_i'$
    Mas a\'un, $G'_i \leq G_{i+1}$ pues $\faktor{G_i}{G_{i+1}}$ es abeliano y
    $G^{(i+1)} \leq G_{i+1}$. Por lo tanto existe una $n \geq 0$ tal que
    $G^{(n)}=\langle e \rangle$, lo que hace $G$ resoluble.
\end{proof}

\begin{example}\label{}
    \begin{enumerate}
        \item[(1)] $D_8$ es resoluble. Escoja $\langle e \rangle \unlhd
            \langle r^4 \rangle \unlhd \langle r^2 \rangle D_8$.

        \item[(2)] Tenemos la serie subnormal $\langle e \rangle \unlhd C_2
            \times C_2 \unlhd A_4 \unlhd S_4$. Donde $C_2 \times C_2=\{(1), (1 \
            2)(3 \ 4), (1 \ 3)(2 \ 4)\}$ Nota que $\faktor{C_2 \times
            C_2}{\langle e \rangle}=C_2 \times C_2 \simeq V_4$, que
            $\faktor{A_4}{C_2 \times C_2} \simeq \faktor{\Z}{3\Z}$ y
            $\faktor{S_4}{A_4} \simeq \faktor{\Z}{2\Z}$. As\'i que $S_4$ es
            resoluble.
    \end{enumerate}
\end{example}
